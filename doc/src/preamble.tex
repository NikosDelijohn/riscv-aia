
%------------------------------------------------------------------------------
% Packages.
%------------------------------------------------------------------------------

\usepackage{graphicx}
\usepackage{colortbl}
\usepackage[svgnames]{xcolor}

\usepackage[colorlinks,citecolor=Navy,linkcolor=Navy]{hyperref}
\usepackage{placeins}
\usepackage{enumitem}
\usepackage{amsmath}

%------------------------------------------------------------------------------
% Main parameters.
%------------------------------------------------------------------------------

\setlength{\topmargin}{-0.5in}
\setlength{\textheight}{9in}
\setlength{\oddsidemargin}{0in}
\setlength{\evensidemargin}{0in}
\setlength{\textwidth}{6.5in}

\pagestyle{myheadings}
\setlength{\parindent}{0in}
\setlength{\parskip}{10pt}
\sloppy
\raggedbottom
\clubpenalty=10000
\widowpenalty=10000

\newlength{\defaultListTopsep}
\defaultListTopsep=0.25\parskip
\setlist{
	partopsep=0mm,
	topsep=\defaultListTopsep,
	itemsep=0mm,
	parsep=\parskip
}

\intextsep=2.5\parskip
\textfloatsep=2.5\parskip
\abovecaptionskip=0.25\parskip
\belowcaptionskip=-0.25\parskip

\renewcommand{\footnoterule}{%
	\vfill\kern\parskip
	\hrule width \textwidth height 1pt
	\kern 0.25\baselineskip
}

%------------------------------------------------------------------------------
% Custom list and table environments.
%------------------------------------------------------------------------------

\newlist{tightList}{itemize}{1}
\setlist[tightList]{label=\textbullet, nosep}

\newenvironment{displayLinesTable}[1][l]{%
	\par
	\vspace{\defaultListTopsep}%
	\noindent\begin{tabular}{@{\hspace{\leftmargin}}#1}%
}{%
	\end{tabular}\par
	\vspace{\defaultListTopsep}%
	\ignorespacesafterend
}

\newenvironment{commentary}{%
	\vspace{-1.5mm}
	\list{}{
		\topsep		0mm
		\partopsep	0mm
		\listparindent	1.5em
		\itemindent	\listparindent
		\rightmargin	\leftmargin
		\parsep		0mm
	}
	\item
	\small\em
	\noindent\nopagebreak\rule{\linewidth}{1pt}\par
	\noindent\ignorespaces
}{%
	\endlist
}

%------------------------------------------------------------------------------
% Commands for register format figures.
%------------------------------------------------------------------------------

% New column types to use in tabular environment for instruction formats.
% Allocate 0.18in per bit.
\newcolumntype{I}{>{\centering\arraybackslash}p{0.18in}}
% Two-bit centered column.
\newcolumntype{W}{>{\centering\arraybackslash}p{0.36in}}
% Three-bit centered column.
\newcolumntype{F}{>{\centering\arraybackslash}p{0.54in}}
% Four-bit centered column.
\newcolumntype{Y}{>{\centering\arraybackslash}p{0.72in}}
% Five-bit centered column.
\newcolumntype{R}{>{\centering\arraybackslash}p{0.9in}}
% Six-bit centered column.
\newcolumntype{S}{>{\centering\arraybackslash}p{1.08in}}
% Seven-bit centered column.
\newcolumntype{O}{>{\centering\arraybackslash}p{1.26in}}
% Eight-bit centered column.
\newcolumntype{E}{>{\centering\arraybackslash}p{1.44in}}
% Ten-bit centered column.
\newcolumntype{T}{>{\centering\arraybackslash}p{1.8in}}
% Twelve-bit centered column.
\newcolumntype{M}{>{\centering\arraybackslash}p{2.2in}}
% Sixteen-bit centered column.
\newcolumntype{K}{>{\centering\arraybackslash}p{2.88in}}
% Twenty-bit centered column.
\newcolumntype{U}{>{\centering\arraybackslash}p{3.6in}}
% Twenty-bit centered column.
\newcolumntype{L}{>{\centering\arraybackslash}p{3.6in}}
% Twenty-five-bit centered column.
\newcolumntype{J}{>{\centering\arraybackslash}p{4.5in}}

\newcommand{\instbit}[1]{\mbox{\scriptsize #1}}
\newcommand{\instbitrange}[2]{~\instbit{#1} \hfill \instbit{#2}~}
\newcommand{\reglabel}[1]{\hfill {\tt #1}\hfill\ }

%------------------------------------------------------------------------------
%------------------------------------------------------------------------------

\newcommand{\z}[1]{{\tt\catcode`\`=\active\protect\frenchspacing#1}}
\newcommand{\zSafe}[1]{{\tt #1}}
\newcommand{\LB}{\char123}
\newcommand{\RB}{\char125}
\newcommand{\RISCV}{\mbox{RISC-V}}

\newcommand{\WIRI}{\textbf{WIRI}}
\newcommand{\WPRI}{\textbf{WPRI}}
\newcommand{\WLRL}{\textbf{WLRL}}
\newcommand{\WARL}{\textbf{WARL}}

\newcommand{\unspecified}{\textsc{unspecified}}



%=======================================================================
\chapter{Duo-PLIC}
\label{ch:DuoPLIC}

\textbf{%
Warning!
This draft specification is likely to change before being accepted as
standard by the {\RISCV} International Association.%
}
\bigskip

The Advanced PLIC of the previous chapter is not backward compatible
with the original {\RISCV} PLIC.%
\footnote{%
A proposed standard for the original PLIC is the document
\textit{{\RISCV} Platform-Level Interrupt Controller Specification},
available as of mid-2021 here:
\z{https://github.com/{\allowbreak}riscv/{\allowbreak}riscv-plic-spec}.
However, both free implementations of the PLIC and those of some major
{\RISCV} vendors such as SiFive have been observed to differ is small
but significant ways from this document and from each other.
Some important details of the original PLIC were better documented in
older versions of
\textit{The {\RISCV} Instruction Set Manual, Volume~II: Privileged
Architecture}, in a chapter titled
``Platform-Level Interrupt Controller (PLIC).''
That chapter has since been removed from more recent versions of
\textit{The {\RISCV} Instruction Set Manual}.%
}
To ease the transition of software from the original PLIC to the
Advanced PLIC, a {\RISCV} system may have a \emph{\mbox{Duo-PLIC\/}}
that software can configure to act as either form of PLIC.
A \mbox{Duo-PLIC} is an extension of an Advanced PLIC, having all the
usual memory-mapped interfaces of an Advanced PLIC, plus, at some other
memory addresses, the memory-mapped interface of an original {\RISCV}
PLIC too.
When the \mbox{Duo-PLIC} has \emph{compatibility mode} enabled,
interrupts from the \mbox{Duo-PLIC} to harts are controlled exclusively
by the original PLIC side of the \mbox{Duo-PLIC};
and when compatibility mode is disabled, interrupts to harts
are controlled exclusively by the Advanced PLIC side of the
\mbox{Duo-PLIC}.
A \mbox{Duo-PLIC} comes out of reset with compatibility mode enabled,
so older software with no knowledge of an Advanced PLIC can see the
\mbox{Duo-PLIC} as an original PLIC.

When compatibility mode is enabled so a \mbox{Duo-PLIC} acts as an
original PLIC, the \mbox{Duo-PLIC} supports only the direct delivery of
interrupts to harts, not the forwarding of interrupts by MSIs.
When compatibility mode is disabled and the \mbox{Duo-PLIC} acts as
an Advanced PLIC, direct delivery mode is supported for all interrupt
domains, and MSI delivery mode may optionally be supported.
Hence, for each Advanced PLIC interrupt domain, field DM
(Delivery Mode) of the \z{domaincfg} register is either read-only zero
or writable, but is not read-only one.

In the memory-mapped control region for the Advanced PLIC root
interrupt domain (only), a \mbox{Duo-PLIC} adds to the \z{domaincfg}
register a CM (Compatibility Mode) field in bit~6.
The complete format of the root interrupt domain's \z{domaincfg} is
thus:
\begin{displayLinesTable}[l@{\quad}l]
bits 31:24 & read-only \z{0x80} \\
bit 8      & IE \\
bit 7      & read-only 0 \\
bit 6      & CM \\
bit 2      & DM (\WARL) \\
bit 0      & BE (\WARL) \\
\end{displayLinesTable}
All other register bits are reserved and read as zeros.

Reset initializes the Advanced PLIC root domain's \z{domaincfg}
register with CM =~1 and with all other writable bits zeros.

When CM~=~1 (compatibility mode enabled):
\begin{itemize}

\item
For each \emph{context} implemented by the original PLIC side of the
\mbox{Duo-PLIC}, the output interrupt for that context becomes the
interrupt signal delivered from the \mbox{Duo-PLIC} to the specific
hart and privilege level specified by the implementation.
Other than the CM field of the Advanced PLIC root domain's
\z{domaincfg} register, the state of the Advanced PLIC side of the
\mbox{Duo-PLIC} is irrelevant to the interrupts delivered to harts.

\item
In the memory-mapped control region for each Advanced PLIC interrupt
domain:
\begin{itemize}

\item
Fields IE and DM of the \z{domaincfg} register are read-only zeros.

\item
Field BE of the \z{domaincfg} register has its usual function for the
interrupt domain but has no effect on accesses to the registers of the
original PLIC side of the \mbox{Duo-PLIC}.

\item
If they exist, the registers that configure MSI addresses
(\z{mmsiaddrcfg}, \z{mmsiaddrcfgh}, \z{smsiaddrcfg}, and
\z{smsiaddrcfgh}) remain functional.

\item
All other registers are read-only zeros.
Writes to the registers are ignored.

\end{itemize}

\end{itemize}

When CM~=~0 (compatibility mode disabled):
\begin{itemize}

\item
Interrupts to harts are determined by the Advanced PLIC side of the
\mbox{Duo-PLIC}, as specified in Chapter~\ref{ch:AdvPLIC}.
The state of the original PLIC side of the \mbox{Duo-PLIC} is
irrelevant.

\item
In the memory-mapped control region for the original PLIC side of the
\mbox{Duo-PLIC}, all registers are read-only zeros.
Writes to the registers are ignored.

\end{itemize}

If the \mbox{Duo-PLIC}'s compatibility mode is changed from disabled to
enabled, the registers of the original PLIC side of the \mbox{Duo-PLIC}
obtain valid and consistent but otherwise {\unspecified} values.

If the \mbox{Duo-PLIC}'s compatibility mode is changed from
enabled to disabled, the registers of the Advanced PLIC side of
the \mbox{Duo-PLIC} obtain valid and consistent but otherwise
{\unspecified} values, except for the \z{domaincfg} registers
and, if they exist, the registers that configure MSI addresses
(\z{mmsiaddrcfg}, \z{mmsiaddrcfgh}, \z{smsiaddrcfg}, and
\z{smsiaddrcfgh}), all of which retain their previous values.
(The root domain's \z{domaincfg} gets the value written when changing
field~CM, of course.)

The endianness of the memory-mapped registers of the original PLIC side
of the \mbox{Duo-PLIC} is not affected by the BE field of any Advanced
PLIC \z{domaincfg} register.
For most, maybe all versions of an original PLIC, the registers are
always little-endian.

The number of interrupt priority levels supported by the original PLIC
side of a \mbox{Duo-PLIC} is the same as that of the Advanced PLIC side
when in direct delivery mode.
Hence, the number of integer bits to store a priority number, IPRIOLEN,
is a constant for a \mbox{Duo-PLIC} (in the range 1 to 8 as usual),
independent of whether compatibility mode is enabled or disabled.

However, an original PLIC encodes priorities in the opposite order
from an IMSIC or Advanced PLIC, with number 1 being interpreted as the
lowest priority and larger numbers as higher priorities.
When a \mbox{Duo-PLIC} is acting as an original PLIC and reports to
a hart an interrupt priority number (see Chapter~\ref{ch:MSLevel}
concerning a hart's handling of interrupt priorities), the number
reported is negated to \emph{normalize} it to the convention of the
Advanced Interrupt Architecture.
If the original PLIC's priority number for an interrupt is~$p$,
the normalized priority number communicated to the target hart is
${\mbox{2}^\textrm{IPRIOLEN}-p}$.

\begin{commentary}
Negation of original PLIC priority numbers to normalize them matters
only when a hart supports configurability of the priorities of major
interrupts, an option added by the Advanced Interrupt Architecture.
See Sections \ref{sec:intrPrios-M} and~\ref{sec:intrPrios-S}.
Software that is oblivious to the Advanced Interrupt Architecture
should have no reason to see priority numbers being negated.
\end{commentary}



%=======================================================================
\chapter{Control and Status Registers (CSRs) Added to Harts}
\label{ch:CSRs}
\chaptermark{CSRs Added to Harts}

For each privilege level at which a {\RISCV} hart can take interrupt
traps, the Advanced Interrupt Architecture adds CSRs for interrupt
control and handling.

%-----------------------------------------------------------------------
\section{Machine-level CSRs}

Table~\ref{tab:CSRs-M} lists both the CSRs added for machine level
and existing machine-level CSRs whose size is changed
by the Advanced Interrupt Architecture.
Existing CSRs \z{mie}, \z{mip}, and \z{mideleg} are
widended to 64~bits to support a total of 64 interrupt causes.

\begin{table*}[h!]
\begin{center}
\begin{tabular}{|c|c|c|l|l|}
\hline
Number & Privilege & Width & Name & Description \\
\hline
\hline
\multicolumn{5}{|c|}{Machine-Level Window to Indirectly Accessed Registers} \\
\hline
\z{0x350} & MRW & XLEN  & \z{miselect} & Machine indirect register select \\
\z{0x351} & MRW & XLEN  & \z{mireg}    & Machine indirect register alias \\
\hline
\multicolumn{5}{|c|}{Machine-Level Interrupts} \\
\hline
\z{0x304} & MRW & 64    & \z{mie}    & Machine interrupt-enable bits \\
\z{0x344} & MRW & 64    & \z{mip}    & Machine interrupt-pending bits \\
\z{0x35C} & MRW & MXLEN & \z{mtopei}
                              & Machine top external interrupt (only with an \\
          &     &       &            & \quad IMSIC) \\
\z{0xFB0} & MRO & MXLEN & \z{mtopi}  & Machine top interrupt \\
\hline
\multicolumn{5}{|c|}{Delegated and Virtual Interrupts for Supervisor Level} \\
\hline
\z{0x303} & MRW & 64 & \z{mideleg} & Machine interrupt delegation \\
\z{0x308} & MRW & 64 & \z{mvien}   & Machine virtual interrupt enables \\
\z{0x309} & MRW & 64 & \z{mvip}    & Machine virtual interrupt-pending bits \\
\hline
\multicolumn{5}{|c|}{Machine-Level High-Half CSRs (RV32 only)} \\
\hline
\z{0x313} & MRW & 32 & \z{midelegh}
                        & Upper 32 bits of of \z{mideleg} (only with S-mode) \\
\z{0x314} & MRW & 32 & \z{mieh} & Upper 32 bits of \z{mie} \\
\z{0x318} & MRW & 32 & \z{mvienh}
                             & Upper 32 bits of \z{mvien} (only with S-mode) \\
\z{0x319} & MRW & 32 & \z{mviph}
                              & Upper 32 bits of \z{mvip} (only with S-mode) \\
\z{0x354} & MRW & 32 & \z{miph} & Upper 32 bits of \z{mip} \\
\hline
\end{tabular}
\end{center}
\caption{%
Machine-level CSRs added or widened by the Advanced Interrupt Architecture.%
}
\label{tab:CSRs-M}
\end{table*}

For RV32, the \emph{high-half} CSRs listed in the table
allow access to the upper 32~bits of
registers \z{mideleg}, \z{mie}, \z{mvien}, \z{mvip}, and \z{mip}.
The Advanced Interrupt Architecture requires that these high-half CSRs
exist for RV32, but the bits they access may all be merely read-only zeros.

CSRs \z{miselect} and \z{mireg} provide a window for accessing multiple
registers beyond the CSRs in Table~\ref{tab:CSRs-M}.
The value of \z{miselect} determines which register is currently
accessible through alias CSR \z{mireg}.
\z{miselect} is a {\WARL} register, and it must support a minimum range
of values depending on the implemented features.
When an IMSIC is not implemented, \z{miselect} must be able to hold at
least any \mbox{6-bit} value in the range 0 to \z{0x3F}.
When an IMSIC is implemented, \z{miselect} must be able to hold any
\mbox{8-bit} value in the range 0 to \z{0xFF}.
Values for \z{miselect} in the range 0 to \z{0xFF} are currently
assigned in subranges as follows:\nopagebreak
\begin{displayLinesTable}[l@{\quad}l]
\z{0x00}--\z{0x2F} & reserved \\
\z{0x30}--\z{0x3F} & major interrupt priorities \\
\z{0x40}--\z{0x6F} & reserved \\
\z{0x70}--\z{0xFF} & external interrupts (only with an IMSIC) \\
\end{displayLinesTable}
\z{miselect} may also support values outside the range
\z{0x00}--\z{0xFF}, though no standard registers are currently
allocated to values above \z{0xFF}.

Values of \z{miselect} with the most-significant bit set
(bit $\mbox{XLEN - 1} = \mbox{1}$) are designated for custom use,
presumably for accessing custom registers through \z{mireg}.
If XLEN changes, the most-significant bit of \z{miselect} moves to
the new position, retaining its value from before.
An implementation is not required to support any custom values for
\z{miselect}.

When \z{miselect} is a number in a reserved range (currently
\z{0x00}--\z{0x2F}, \z{0x40}--\z{0x6F}, or a number above \z{0xFF}
not designated for custom use), attempts to access \z{mireg}
will typically raise an
illegal instruction exception.

Normally, the range for external interrupts, \z{0x70}--\z{0xFF}, is
populated only when an IMSIC is implemented; else, attempts to access
\z{mireg} when \z{miselect} is in this range also cause an illegal
instruction exception.
The contents of the external-interrupts region are documented in
Chapter~\ref{ch:IMSIC} on the IMSIC.

CSR \z{mtopei} also exists only when an IMSIC is implemented,
so is documented in Chapter~\ref{ch:IMSIC}
along with the indirectly accessed IMSIC registers.

CSR \z{mtopi} reports the highest-priority interrupt that is pending
and enabled for machine level, as specified in Section~\ref{sec:mtopi}.

When \mbox{S-mode} is implemented, CSRs \z{mvien} and \z{mvip}
support interrupt
filtering and virtual interrupts for supervisor level.
These facilities are explained in Section~\ref{sec:virtIntrs-S}.

If extension Smcsrind is also implemented, then when \z{miselect}
has a value in the range \z{0x30}--\z{0x3F} or \z{0x70}--\z{0xFF},
attempts to access alias CSRs \z{mireg2} through \z{mireg6}
raise an illegal instruction exception.

%-----------------------------------------------------------------------
\section{Supervisor-level CSRs}

Table~\ref{tab:CSRs-S} lists the supervisor-level CSRs that are added
and existing CSRs that are widened to 64~bits,
if the hart implements \mbox{S-mode}.
The functions of these registers all match their machine-level
counterparts.

\begin{table*}[h!]
\begin{center}
\begin{tabular}{|c|c|c|l|l|}
\hline
Number & Privilege & Width & Name      & Description \\
\hline
\hline
\multicolumn{5}{|c|}{%
  Supervisor-Level Window to Indirectly Accessed Registers} \\
\hline
\z{0x150} & SRW & XLEN  & \z{siselect} & Supervisor indirect register select \\
\z{0x151} & SRW & XLEN  & \z{sireg}    & Supervisor indirect register alias \\
\hline
\multicolumn{5}{|c|}{Supervisor-Level Interrupts} \\
\hline
\z{0x104} & SRW & 64    & \z{sie}      & Supervisor interrupt-enable bits \\
\z{0x144} & SRW & 64    & \z{sip}      & Supervisor interrupt-pending bits \\
\z{0x15C} & SRW & SXLEN & \z{stopei}
                                   & Supervisor top external interrupt (only \\
          &     &       &          & \quad with an IMSIC) \\
\z{0xDB0} & SRO & SXLEN & \z{stopi}    & Supervisor top interrupt \\
\hline
\multicolumn{5}{|c|}{Supervisor-Level High-Half CSRs (RV32 only)} \\
\hline
\z{0x114} & SRW & 32    & \z{sieh}     & Upper 32 bits of \z{sie} \\
\z{0x154} & SRW & 32    & \z{siph}     & Upper 32 bits of \z{sip} \\
\hline
\end{tabular}
\end{center}
\caption{%
Supervisor-level CSRs added or widened by the Advanced Interrupt Architecture.%
}
\label{tab:CSRs-S}
\end{table*}

The space of registers accessible through the \z{siselect}/\z{sireg}
window is separate from but parallels that of machine level, being for
supervisor-level interrupts instead of machine-level interrupts.
The allocated values for \z{siselect} in the range 0 to \z{0xFF} are
once again these:\nopagebreak
\begin{displayLinesTable}[l@{\quad}l]
\z{0x00}--\z{0x2F} & reserved \\
\z{0x30}--\z{0x3F} & major interrupt priorities \\
\z{0x40}--\z{0x6F} & reserved \\
\z{0x70}--\z{0xFF} & external interrupts (only with an IMSIC) \\
\end{displayLinesTable}
For maximum compatibility, it is recommended that \z{siselect} support
at least a \mbox{9-bit} range, 0 to \z{0x1FF}, regardless of whether an
IMSIC exists.

\begin{commentary}
Because the VS CSR \z{vsiselect} (Section~\ref{ch:CSRs-hypervisor})
always has at least 9~bits, and like other VS CSRs, \z{vsiselect}
substitutes for \z{siselect} when executing in a virtual machine
(\mbox{VS-mode} or \mbox{VU-mode}), implementing a smaller range for
\z{siselect} allows software to discover it is not running in a virtual
machine.
\end{commentary}

Like \z{miselect}, values of \z{siselect} with the most-significant bit
set (bit $\mbox{XLEN - 1} = \mbox{1}$) are designated for custom use.
If XLEN changes, the most-significant bit of \z{siselect} moves to
the new position, retaining its value from before.
An implementation is not required to support any custom values for
\z{siselect}.

When \z{siselect} is a number in a reserved range (currently
\z{0x00}--\z{0x2F}, \z{0x40}--\z{0x6F}, or a number above \z{0xFF}
not designated for custom use), or in the range \z{0x70}--\z{0xFF}
when there is no IMSIC, attempts to access \z{sireg}
should preferably raise an illegal
instruction exception (unless executing in a virtual machine, covered
in the next section).

Note that the widths of \z{siselect} and \z{sireg}
are always the current XLEN rather than SXLEN\@.
Hence, for example, if MXLEN = 64 and SXLEN = 32, then these registers
are 64~bits when the current privilege mode is M (running RV64 code)
but 32~bits when the privilege mode is S (RV32 code).

CSR \z{stopei} is described with the IMSIC in Chapter~\ref{ch:IMSIC}.

Register \z{stopi} reports the highest-priority interrupt that
is pending and enabled for supervisor level, as specified in
Section~\ref{sec:stopi}.

If extension Sscsrind is also implemented, then when \z{siselect}
has a value in the range \z{0x30}--\z{0x3F} or \z{0x70}--\z{0xFF},
attempts to access alias CSRs \z{sireg2} through \z{sireg6}
raise an illegal instruction exception (unless executing
in a virtual machine, covered in the next section).

%-----------------------------------------------------------------------
\section{Hypervisor and VS CSRs}
\label{ch:CSRs-hypervisor}

If a hart implements the Privileged Architecture's hypervisor
extension, then the hypervisor and VS CSRs listed in
Table~\ref{tab:CSRs-hypervisor} are also either added
or widened to 64~bits.

\begin{table*}[h!]
\begin{center}
\begin{tabular}{|c|c|c|l|l|}
\hline
Number & Privilege & Width & Name & Description \\
\hline
\hline
\multicolumn{5}{|c|}{%
  Delegated and Virtual Interrupts, Interrupt Priorities, for VS Level%
} \\
\hline
\z{0x603} & HRW & 64     & \z{hideleg} & Hypervisor interrupt delegation \\
\z{0x608} & HRW & 64     & \z{hvien}  & Hypervisor virtual interrupt enables \\
\z{0x609} & HRW & HSXLEN & \z{hvictl} & Hypervisor virtual interrupt control \\
\z{0x645} & HRW & 64     & \z{hvip}
                                 & Hypervisor virtual interrupt-pending bits \\
\z{0x646} & HRW & 64     & \z{hviprio1}
                                 & Hypervisor VS-level interrupt priorities \\
\z{0x647} & HRW & 64     & \z{hviprio2}
                                 & Hypervisor VS-level interrupt priorities \\
\hline
\multicolumn{5}{|c|}{VS-Level Window to Indirectly Accessed Registers} \\
\hline
\z{0x250} & HRW & XLEN   & \z{vsiselect}
                               & Virtual supervisor indirect register select \\
\z{0x251} & HRW & XLEN   & \z{vsireg}
                               & Virtual supervisor indirect register alias \\
\hline
\multicolumn{5}{|c|}{VS-Level Interrupts} \\
\hline
\z{0x204} & HRW & 64     & \z{vsie}
                                 & Virtual supervisor interrupt-enable bits \\
\z{0x244} & HRW & 64     & \z{vsip}
                                 & Virtual supervisor interrupt-pending bits \\
\z{0x25C} & HRW & VSXLEN & \z{vstopei}
                           & Virtual supervisor top external interrupt (only \\
          &     &        & & \quad with an IMSIC) \\
\z{0xEB0} & HRO & VSXLEN & \z{vstopi} & Virtual supervisor top interrupt \\
\hline
\multicolumn{5}{|c|}{Hypervisor and VS-Level High-Half CSRs (RV32 only)} \\
\hline
\z{0x613} & HRW & 32     & \z{hidelegh}  & Upper 32 bits of \z{hideleg} \\
\z{0x618} & HRW & 32     & \z{hvienh}    & Upper 32 bits of \z{hvien} \\
\z{0x655} & HRW & 32     & \z{hviph}     & Upper 32 bits of \z{hvip} \\
\z{0x656} & HRW & 32     & \z{hviprio1h} & Upper 32 bits of \z{hviprio1} \\
\z{0x657} & HRW & 32     & \z{hviprio2h} & Upper 32 bits of \z{hviprio2} \\
\z{0x214} & HRW & 32     & \z{vsieh}     & Upper 32 bits of \z{vsie} \\
\z{0x254} & HRW & 32     & \z{vsiph}     & Upper 32 bits of \z{vsip} \\
\hline
\end{tabular}
\end{center}
\caption{%
Hypervisor and VS CSRs added or widened
by the Advanced Interrupt Architecture.
(Parameter HSXLEN is just another name for
SXLEN for hypervisor-extended \mbox{S-mode}).%
}
\label{tab:CSRs-hypervisor}
\end{table*}

The new hypervisor CSRs in the table (\z{hvien},
\z{hvictl}, \z{hviprio1}, and \z{hviprio2}) augment
\z{hvip} for injecting interrupts into VS level.
The use of these registers is covered in Chapter~\ref{ch:VSLevel} on
interrupts for virtual machines.

The new VS CSRs (\z{vsiselect}, \z{vsireg},
\z{vstopei}, and \z{vstopi})
all match supervisor CSRs, and substitute for those
supervisor CSRs when executing in a virtual machine (in \mbox{VS-mode}
or \mbox{VU-mode}).

CSR \z{vsiselect} is required to support at least a \mbox{9-bit} range
of 0 to \z{0x1FF}, whether or not an IMSIC is implemented.
As with \z{siselect}, values of \z{vsiselect} with the most-significant
bit set (bit $\mbox{XLEN - 1} = \mbox{1}$) are designated for custom
use.
If XLEN changes, the most-significant bit of \z{vsiselect} moves
to the new position, retaining its value from before.

Like \z{siselect} and \z{sireg}, the widths of \z{vsiselect}
and \z{vsireg} are always the current XLEN rather than VSXLEN\@.
Hence, for example, if HSXLEN = 64 and VSXLEN = 32, then these
registers are 64~bits when accessed by a hypervisor in HS-mode
(running RV64 code) but 32~bits for a guest OS in VS-mode (RV32 code).

The space of registers selectable by \z{vsiselect} is more limited than
for machine and supervisor levels:\nopagebreak
\begin{displayLinesTable}[l@{\quad}l]
\z{0x000}--\z{0x02F} & reserved \\
\z{0x030}--\z{0x03F} & inaccessible \\
\z{0x040}--\z{0x06F} & reserved \\
\z{0x070}--\z{0x0FF} & external interrupts (IMSIC only), or inaccessible \\
\z{0x100}--\z{0x1FF} & reserved \\
\end{displayLinesTable}

For alias CSRs \z{sireg} and \z{vsireg}, the hypervisor extension's
usual rules for when to raise a virtual instruction exception (based
on whether an instruction is \emph{HS-qualified}) are not applicable.
The rules given in this section for \z{sireg} and \z{vsireg}
apply instead, unless overridden by the requirements of
Section~\ref{sec:CSRs-stateen}, which take precedence over
this section when extension Smstateen is also implemented.

A virtual instruction exception is raised for attempts from
\mbox{VS-mode} or \mbox{VU-mode} to directly access \z{vsireg},
or attempts from \mbox{VU-mode} to access \z{sireg}.

When \z{vsiselect} has a reserved value (including values above
\z{0x1FF} not designated for custom use), attempts from \mbox{M-mode}
or \mbox{HS-mode} to access \z{vsireg}, or from \mbox{VS-mode}
to access \z{sireg} (really \z{vsireg}),
should preferably raise an illegal instruction exception.

When \z{vsiselect} has the number of an \emph{inaccessible} register,
attempts from \mbox{M-mode} or \mbox{HS-mode} to access \z{vsireg}
raise an illegal instruction exception, and attempts from
\mbox{VS-mode} to access \z{sireg}
(really \z{vsireg}) raise a virtual instruction exception.

\begin{commentary}
Requiring a range of\/ {\rm 0}--\z{0x1FF} for \z{vsiselect}, even
though most or all of the space is
reserved or inaccessible, permits a hypervisor
to emulate indirectly accessed registers in the implemented range,
including registers that may be standardized in the future at locations
\z{0x100}--\z{0x1FF}.
\end{commentary}

The indirectly accessed registers for external interrupts (numbers
\z{0x70}--\z{0xFF}) are accessible only when field VGEIN of \z{hstatus}
is the number of an implemented guest external interrupt, not zero.
If VGEIN is not the number of an implemented guest external interrupt
(including the case when no IMSIC is implemented), then all
indirect register numbers in the ranges \z{0x030}--\z{0x03F} and
\z{0x070}--\z{0x0FF} designate an inaccessible register at VS level.

Along the same lines, when \z{hstatus}.VGEIN is not the number of
an implemented guest external interrupt, attempts from \mbox{M-mode}
or \mbox{HS-mode} to access CSR \z{vstopei}
raise an illegal instruction exception,
and attempts from \mbox{VS-mode} to access \z{stopei} raise a
virtual instruction exception.

If extension Sscsrind is also implemented, then when \z{vsiselect} has
a value in the range \z{0x30}--\z{0x3F} or \z{0x70}--\z{0xFF},
attempts from \mbox{M-mode} or \mbox{HS-mode} to access alias CSRs
\z{vsireg2} through \z{vsireg6} raise an illegal instruction exception,
and attempts from \mbox{VS-mode} to access \z{sireg2} through \z{sireg6}
raise a virtual instruction exception.

%-----------------------------------------------------------------------
\section{Virtual instruction exceptions}

Following the default rules for the hypervisor extension, attempts
from \mbox{VS-mode} to directly access a hypervisor or VS CSR
other than \z{vsireg}, or from \mbox{VU-mode} to access any
supervisor-level CSR (including hypervisor and VS CSRs) other than
\z{sireg} or \z{vsireg}, usually raise not an illegal instruction
exception but instead a virtual instruction exception.
For details, see the {\RISCV} Privileged Architecture.

Instructions that read/write CSR \z{stopei} or \z{vstopei} are
considered to be \emph{HS-qualified} unless all of following are true:
the hart has an \mbox{IMSIC}, extension Smstateen is
implemented, and bit~58 of \z{mstateen0} is zero.
(See the next section, \ref{sec:CSRs-stateen}, about \z{mstateen0}.)

For \z{sireg} and \z{vsireg}, see both the previous section,
\ref{ch:CSRs-hypervisor}, and the next, \ref{sec:CSRs-stateen},
for when a virtual instruction exception is required
instead of an illegal instruction exception.

%-----------------------------------------------------------------------
\section{Access control by the state-enable CSRs}
\label{sec:CSRs-stateen}

If extension Smstateen is implemented together with the
Advanced Interrupt Architecture (AIA), three bits of state-enable
register \z{mstateen0} control access to AIA-added state
from privilege modes less privileged than \mbox{M-mode}:\nopagebreak
\begin{displayLinesTable}[l@{\quad}l]
bit 60 & CSRs \z{siselect}, \z{sireg}, \z{vsiselect}, and \z{vsireg} \\
bit 59 & all other state added by the AIA
          and not controlled by bits 60 and 58 \\
bit 58 & all IMSIC state, including CSRs \z{stopei} and \z{vstopei} \\
\end{displayLinesTable}

If one of these bits is zero in
\z{mstateen0}, an attempt to access
the corresponding state from a privilege mode less privileged
than \mbox{M-mode} results in an illegal instruction trap.
As always, the state-enable CSRs do not affect the accessibility
of any state when in \mbox{M-mode}, only in less privileged modes.
For more explanation, see the documentation for extension Smstateen.

Bit~59 controls access to AIA CSRs \z{siph}, \z{sieh}, \z{stopi},
\z{hidelegh}, \z{hvien}/\z{hvienh}, \z{hviph}, \z{hvictl},
\z{hviprio1}/\z{hviprio1h}, \z{hviprio2}/\z{hviprio2h}, \z{vsiph},
\z{vsieh}, and \z{vstopi}, as well as to the supervisor-level interrupt
priorities accessed through \z{siselect} + \z{sireg}
(the \z{iprio} array of Section~\ref{sec:intrPrios-S}).

Bit~58 is implemented in \z{mstateen0}
only if the hart has an IMSIC.
If the hypervisor extension is also implemented, this bit does
not affect the behavior or accessibility of hypervisor CSRs
\z{hgeip} and \z{hgeie}, or field VGEIN of \z{hstatus}.
In particular, guest external interrupts from an IMSIC
continue to be visible to \mbox{HS-mode} in \z{hgeip}
even when bit~58 of \z{mstateen0} is zero.

\begin{commentary}
An earlier, pre-ratification draft of Smstateen said that when
bit~58 of \z{mstateen0} is zero, registers \z{hgeip} and \z{hgeie}
and field VGEIN of \z{hstatus} are all read-only zeros.
That effect is no longer correct.
\end{commentary}

If the hart does not have an \mbox{IMSIC}, bit~58 of \z{mstateen0}
is read-only zero, but Smstateen has no effect on
attempts to access the nonexistent \mbox{IMSIC} state.

\begin{commentary}
This means in particular that, when the hart does not have an
\mbox{IMSIC}, the following raise a virtual instruction exception
as described in Section~\ref{ch:CSRs-hypervisor}, not an illegal
instruction exception, despite that bit~58 of\/ \z{mstateen0} is zero:
\begin{tightList}
\item
attempts from \mbox{VS-mode} to access\/ \z{sireg}
(really\/ \z{vsireg}) while\/ \z{vsiselect} has
a value in the range\/ \z{0x70}--\z{0xFF}; and
\item
attempts from \mbox{VS-mode} to access\/
\z{stopei} (really\/ \z{vstopei}).
\end{tightList}
\end{commentary}

If bit~60 of \z{mstateen0} is one, then regardless of any other
\z{mstateen} bits (including bits 58 and 59 of \z{mstateen0}),
a virtual instruction exception is raised as described in
Section~\ref{ch:CSRs-hypervisor} for all attempts from
\mbox{VS-mode} or \mbox{VU-mode} to directly access \z{vsireg},
and for all attempts from \mbox{VU-mode} to access \z{sireg}.
This behavior is overridden only when bit~60 of \z{mstateen0} is zero.

If the hypervisor extension is implemented, the same three bits are
defined also in hypervisor CSR \z{hstateen0}
but concern only the state potentially
accessible to a virtual machine executing
in privilege modes VS and VU:\nopagebreak
\begin{displayLinesTable}[l@{\quad}l]
bit 60 & CSRs \z{siselect} and \z{sireg}
          (really \z{vsiselect} and \z{vsireg}) \\
bit 59 & CSRs \z{siph} and \z{sieh} (RV32 only) and \z{stopi}
          (really \z{vsiph}, \z{vsieh}, and \z{vstopi}) \\
bit 58 & all state of IMSIC guest interrupt files,
          including CSR \z{stopei} (really \z{vstopei}) \\
\end{displayLinesTable}

If one of these bits is zero in \z{hstateen0},
and the same bit is one in \z{mstateen0},
then an attempt to access the corresponding state
from VS or \mbox{VU-mode} raises a virtual instruction exception.
(But note that, for high-half CSRs \z{siph} and
\z{sieh}, this applies only when XLEN =~32.
When $\mbox{XLEN} > \mbox{32}$, an attempt to access
\z{siph} or \z{sieh} raises an illegal instruction
exception as usual, not a virtual instruction exception.)

If bit~60 is one in \z{mstateen0} but is zero in \z{hstateen0},
then all attempts from VS or \mbox{VU-mode} to access
\z{siselect} or \z{sireg} raise a virtual instruction
exception, not an illegal instruction exception, regardless
of the value of \z{vsiselect} or any other \z{mstateen} bits.

Bit~58 is implemented in \z{hstateen0}
only if the hart has an IMSIC.
Furthermore, even with an IMSIC, bit~58 may (or may not) be read-only
zero in \z{hstateen0} if
the IMSIC has no \emph{guest interrupt files}
for guest external interrupts (Chapter~\ref{ch:IMSIC}).
When this bit is zero (whether read-only zero or set to zero),
a virtual machine is prevented from
accessing the hart's IMSIC the same as when \z{hstatus}.VGEIN =~0.

Extension Ssstateen is defined as
the supervisor-level view of Smstateen.
Therefore, the combination of Ssaia and Ssstateen incorporates the bits
defined above for \z{hstateen0} but not those for \z{mstateen0},
since machine-level CSRs are not visible to supervisor level.


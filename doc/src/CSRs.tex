
%=======================================================================
\chapter{Control and Status Registers (CSRs) Added to Harts}
\label{ch:CSRs}
\chaptermark{CSRs Added to Harts}

\textbf{%
Warning!
This draft specification is likely to change before being accepted as
standard by the {\RISCV} International Association.%
}
\bigskip

For each privilege level at which a {\RISCV} hart can take interrupt
traps, the Advanced Interrupt Architecture adds CSRs for interrupt
control and handling.

\begin{commentary}
The addresses for all CSRs in this document have been carefully
selected not to conflict with existing allocations, or with other
likely extensions, so far as is known.
\end{commentary}

%-----------------------------------------------------------------------
\section{Machine-level CSRs}

Table~\ref{tab:CSRs-M} lists the CSRs added for machine level.

\begin{table*}[h!]
\begin{center}
\begin{tabular}{|l|l|l|l|}
\hline
Number & Privilege & Name      & Description \\
\hline
\hline
\multicolumn{4}{|c|}{Machine-Level Window to Indirectly Accessed Registers} \\
\hline
\z{0x350} & MRW & \z{miselect} & Machine indirect register select \\
\z{0x351} & MRW & \z{mireg}    & Machine indirect register alias \\
\hline
\multicolumn{4}{|c|}{Machine-Level Interrupts} \\
\hline
\z{0xFB0} & MRO & \z{mtopi}    & Machine top interrupt \\
\hline
\multicolumn{4}{|c|}{Machine-Level IMSIC Interface (only with an IMSIC)} \\
\hline
\z{0x358} & MRW & \z{mseteipnum}
                   & Machine set external interrupt-pending bit by number \\
\z{0x359} & MRW & \z{mclreipnum}
                   & Machine clear external interrupt-pending bit by number \\
\z{0x35A} & MRW & \z{mseteienum}
                   & Machine set external interrupt-enable bit by number \\
\z{0x35B} & MRW & \z{mclreienum}
                   & Machine clear external interrupt-enable bit by number \\
\z{0x35C} & MRW & \z{mtopei}   & Machine top external interrupt \\
\hline
\multicolumn{4}{|c|}{Virtual Interrupts for Supervisor Level} \\
\hline
\z{0x308} & MRW & \z{mvien}    & Machine virtual interrupt enables \\
\z{0x309} & MRW & \z{mvip}     & Machine virtual interrupt-pending bits \\
\hline
\multicolumn{4}{|c|}{%
  Machine-Level High-Half CSRs for Interrupts 32--63 (RV32 only)} \\
\hline
\z{0x313} & MRW & \z{midelegh} & Extension of \z{mideleg} (only with S-mode) \\
\z{0x314} & MRW & \z{mieh}     & Extension of \z{mie} \\
\z{0x318} & MRW & \z{mvienh}   & Extension of \z{mvien} (only with S-mode) \\
\z{0x319} & MRW & \z{mviph}    & Extension of \z{mvip} (only with S-mode) \\
\z{0x354} & MRW & \z{miph}     & Extension of \z{mip} \\
\hline
\end{tabular}
\end{center}
\caption{Machine-level CSRs added by the Advanced Interrupt Architecture.}
\label{tab:CSRs-M}
\end{table*}

For RV32, the \emph{high-half} CSRs listed in the table extend
registers \z{mideleg}, \z{mie}, \z{mvien}, \z{mvip}, and \z{mip} to
support a total of 64 interrupt causes.
Each low-half register (e.g. \z{mideleg}) covers interrupt numbers
0--31, one bit per interrupt number, while its high-half partner
(\z{midelegh}) covers interrupt numbers 32--63, again one bit per
interrupt number.
For RV64, a single \mbox{64-bit} CSR (\z{mideleg}) covers the same set
of interrupt numbers, 0--63.
The Advanced Interrupt Architecture requires that these high-half CSRs
exist for RV32, but they may all be merely read-only zeros.

CSRs \z{miselect} and \z{mireg} provide a window for accessing multiple
registers beyond the CSRs in Table~\ref{tab:CSRs-M}.
The value of \z{miselect} determines which register is currently
accessible through alias CSR \z{mireg}.
\z{miselect} is a {\WARL} register, and it must support a minimum range
of values depending on the implemented features.
When an IMSIC is not implemented, \z{miselect} must be able to hold at
least any \mbox{6-bit} value in the range 0 to \z{0x3F}.
When an IMSIC is implemented, \z{miselect} must be able to hold any
\mbox{8-bit} value in the range 0 to \z{0xFF}.
Values for \z{miselect} in the range 0 to \z{0xFF} are currently
assigned in subranges as follows:
\begin{displayLinesTable}[l@{\quad}l]
\z{0x00}--\z{0x2F} & reserved \\
\z{0x30}--\z{0x3F} & major interrupt priorities \\
\z{0x40}--\z{0x6F} & reserved \\
\z{0x70}--\z{0xFF} & external interrupts (only with an IMSIC) \\
\end{displayLinesTable}
\z{miselect} may also support values outside the range
\z{0x00}--\z{0xFF}, though no standard registers are currently
allocated to values above \z{0xFF}.

Values of \z{miselect} with the most-significant bit set
(bit $\mbox{MXLEN - 1} = \mbox{1}$) are designated for custom use,
presumably for accessing custom registers through \z{mireg}.
If MXLEN is changed, the most-significant bit of \z{miselect} moves to
the new position, retaining its value from before.
An implementation is not required to support any custom values for
\z{miselect}.

When \z{miselect} is a number in a reserved range (currently
\z{0x00}--\z{0x2F}, \z{0x40}--\z{0x6F}, or a number above \z{0xFF}
not designated for custom use), attempts to access \z{mireg} raise an
illegal instruction exception.

Normally, the range for external interrupts, \z{0x70}--\z{0xFF}, is
populated only when an IMSIC is implemented; else, attempts to access
\z{mireg} when \z{miselect} is in this range also cause an illegal
instruction exception.
The contents of the external-interrupts region are documented in
Chapter~\ref{ch:IMSIC} on the IMSIC.

CSR \z{mtopi} reports the highest-priority interrupt that is pending
and enabled for machine level, as specified in Section~\ref{sec:mtopi}.

Five CSRs, \z{mseteipnum}, \z{mclreipnum}, \z{mseteienum},
\z{mclreienum}, and \z{mtopei}, interact with an IMSIC, so exist only
when an IMSIC is implemented.
Like the indirectly accessed IMSIC registers, these CSRs are documented
in Chapter~\ref{ch:IMSIC}.

When \mbox{S-mode} is implemented, CSRs \z{mvien} and \z{mvip}
(together with their high-half partners for RV32) support interrupt
filtering and virtual interrupts for supervisor level.
These facilities are explained in Section~\ref{sec:virtIntrs-S}.

%-----------------------------------------------------------------------
\section{Supervisor-level CSRs}

Table~\ref{tab:CSRs-S} lists the supervisor-level CSRs that are added
if the hart implements \mbox{S-mode}.
The functions of these registers all match their machine-level
counterparts.

\begin{table*}[h!]
\begin{center}
\begin{tabular}{|l|l|l|l|}
\hline
Number & Privilege & Name      & Description \\
\hline
\hline
\multicolumn{4}{|c|}{%
  Supervisor-Level Window to Indirectly Accessed Registers} \\
\hline
\z{0x150} & SRW & \z{siselect} & Supervisor indirect register select \\
\z{0x151} & SRW & \z{sireg}    & Supervisor indirect register alias \\
\hline
\multicolumn{4}{|c|}{Supervisor-Level Interrupts} \\
\hline
\z{0xDB0} & SRO & \z{stopi}    & Supervisor top interrupt \\
\hline
\multicolumn{4}{|c|}{Supervisor-Level IMSIC Interface (only with an IMSIC)} \\
\hline
\z{0x158} & SRW & \z{sseteipnum}
                 & Supervisor set external interrupt-pending bit by number \\
\z{0x159} & SRW & \z{sclreipnum}
                 & Supervisor clear external interrupt-pending bit by number \\
\z{0x15A} & SRW & \z{sseteienum}
                 & Supervisor set external interrupt-enable bit by number \\
\z{0x15B} & SRW & \z{sclreienum}
                 & Supervisor clear external interrupt-enable bit by number \\
\z{0x15C} & SRW & \z{stopei}   & Supervisor top external interrupt \\
\hline
\multicolumn{4}{|c|}{%
  Supervisor-Level High-Half CSRs for Interrupts 32--63 (RV32 only)} \\
\hline
\z{0x114} & SRW & \z{sieh}     & Extension of \z{sie} \\
\z{0x154} & SRW & \z{siph}     & Extension of \z{sip} \\
\hline
\end{tabular}
\end{center}
\caption{Supervisor-level CSRs added by the Advanced Interrupt Architecture.}
\label{tab:CSRs-S}
\end{table*}

The space of registers accessible through the \z{siselect}/\z{sireg}
window is separate from but parallels that of machine level, being for
supervisor-level interrupts instead of machine-level interrupts.
The allocated values for \z{siselect} in the range 0 to \z{0xFF} are
once again these:
\begin{displayLinesTable}[l@{\quad}l]
\z{0x00}--\z{0x2F} & reserved \\
\z{0x30}--\z{0x3F} & major interrupt priorities \\
\z{0x40}--\z{0x6F} & reserved \\
\z{0x70}--\z{0xFF} & external interrupts (only with an IMSIC) \\
\end{displayLinesTable}
For maximum compatibility, it is recommended that \z{siselect} support
at least a \mbox{9-bit} range, 0 to \z{0x1FF}, regardless of whether an
IMSIC exists.

\begin{commentary}
Because the VS CSR \z{vsiselect} (Section~\ref{ch:CSRs-hypervisor})
always has at least 9~bits, and like other VS CSRs, \z{vsiselect}
substitutes for \z{siselect} when executing in a virtual machine
(\mbox{VS-mode} or \mbox{VU-mode}), implementing a smaller range for
\z{siselect} allows software to discover it is not running in a virtual
machine.
\end{commentary}

Like \z{miselect}, values of \z{siselect} with the most-significant bit
set (bit $\mbox{SXLEN - 1} = \mbox{1}$) are designated for custom use.
If SXLEN is changed, the most-significant bit of \z{siselect} moves to
the new position, retaining its value from before.
An implementation is not required to support any custom values for
\z{siselect}.

When \z{siselect} is a number in a reserved range (currently
\z{0x00}--\z{0x2F}, \z{0x40}--\z{0x6F}, or a number above \z{0xFF}
not designated for custom use), or in the range \z{0x70}--\z{0xFF}
when there is no IMSIC, attempts to access \z{sireg} raise an illegal
instruction exception (unless executing in a virtual machine, covered
in the next section).

Register \z{stopi} reports the highest-priority interrupt that
is pending and enabled for supervisor level, as specified in
Section~\ref{sec:stopi}.
The IMSIC CSRs, \z{sseteipnum}, \z{sclreipnum}, \z{sseteienum},
\z{sclreienum}, and \z{stopei}, are described with the IMSIC in
Chapter~\ref{ch:IMSIC}.

%-----------------------------------------------------------------------
\section{Hypervisor and VS CSRs}
\label{ch:CSRs-hypervisor}

If a hart implements the Privileged Architecture's hypervisor
extension, then the hypervisor and VS CSRs listed in
Table~\ref{tab:CSRs-hypervisor} are also added.

\begin{table*}[h!]
\begin{center}
\begin{tabular}{|l|l|l|l|}
\hline
Number & Privilege & Name        & Description \\
\hline
\hline
\multicolumn{4}{|c|}{%
  Virtual Interrupts and Interrupt Priorities for VS Level} \\
\hline
\z{0x608} & HRW & \z{hvien}      & Hypervisor virtual interrupt enables \\
\z{0x609} & HRW & \z{hvictl}     & Hypervisor virtual interrupt control \\
\z{0x646} & HRW & \z{hviprio1}   & Hypervisor VS-level interrupt priorities \\
\z{0x647} & HRW & \z{hviprio2}   & Hypervisor VS-level interrupt priorities \\
\hline
\multicolumn{4}{|c|}{VS-Level Window to Indirectly Accessed Registers} \\
\hline
\z{0x250} & HRW & \z{vsiselect}
                   & Virtual supervisor indirect register select \\
\z{0x251} & HRW & \z{vsireg}
                   & Virtual supervisor indirect register alias \\
\hline
\multicolumn{4}{|c|}{VS-Level Interrupts} \\
\hline
\z{0xEB0} & HRO & \z{vstopi}     & Virtual supervisor top interrupt \\
\hline
\multicolumn{4}{|c|}{VS-Level IMSIC Interface (only with an IMSIC)} \\
\hline
\z{0x258} & HRW & \z{vsseteipnum}
                   & VS set external interrupt-pending bit by number \\
\z{0x259} & HRW & \z{vsclreipnum}
                   & VS clear external interrupt-pending bit by number \\
\z{0x25A} & HRW & \z{vsseteienum}
                   & VS set external interrupt-enable bit by number \\
\z{0x25B} & HRW & \z{vsclreienum}
                   & VS clear external interrupt-enable bit by number \\
\z{0x25C} & SRW & \z{vstopei}    & Virtual supervisor top external interrupt \\
\hline
\multicolumn{4}{|c|}{Hypervisor and VS-Level High-Half CSRs (RV32 only)} \\
\hline
\z{0x613} & HRW & \z{hidelegh}   & Extension of \z{hideleg} \\
\z{0x618} & HRW & \z{hvienh}     & Extension of \z{hvien} \\
\z{0x655} & HRW & \z{hviph}      & Extension of \z{hvip} \\
\z{0x656} & HRW & \z{hviprio1h}  & Extension of \z{hviprio1} \\
\z{0x657} & HRW & \z{hviprio2h}  & Extension of \z{hviprio2} \\
\z{0x214} & HRW & \z{vsieh}      & Extension of \z{vsie} \\
\z{0x254} & HRW & \z{vsiph}      & Extension of \z{vsip} \\
\hline
\end{tabular}
\end{center}
\caption{Hypervisor and VS CSRs added by the Advanced Interrupt Architecture.}
\label{tab:CSRs-hypervisor}
\end{table*}

The VS CSRs in the table (\z{vsiselect}, \z{vsireg}, \z{vstopi},
\z{vsseteipnum}, \z{vsclreipnum}, \z{vsseteienum}, and
\z{vsclreienum}) all match supervisor CSRs, and substitute for those
supervisor CSRs when executing in a virtual machine (in \mbox{VS-mode}
or \mbox{VU-mode}).

CSR \z{vsiselect} is required to support at least a \mbox{9-bit} range
of 0 to \z{0x1FF}, whether or not an IMSIC is implemented.
As with \z{siselect}, values of \z{vsiselect} with the most-significant
bit set (bit $\mbox{VSXLEN - 1} = \mbox{1}$) are designated for custom
use.
If VSXLEN is changed, the most-significant bit of \z{vsiselect} moves
to the new position, retaining its value from before.

The space of registers selectable by \z{vsiselect} is more limited than
for machine and supervisor levels:
\begin{displayLinesTable}[l@{\quad}l]
\z{0x000}--\z{0x06F} & inaccessible \\
\z{0x070}--\z{0x0FF} & external interrupts (IMSIC only), or inaccessible \\
\z{0x100}--\z{0x1FF} & inaccessible \\
\end{displayLinesTable}
When \z{vsiselect} has the number of an \emph{inaccessible} register,
including values above \z{0x1FF} without the most-significant bit set,
attempts from \mbox{M-mode} or \mbox{HS-mode} to access \z{vsireg}
raise an illegal instruction exception, and attempts from
\mbox{VS-mode} to access \z{sireg} (really \z{vsireg}) raise a virtual
instruction exception.

\begin{commentary}
Requiring a range of\/ {\rm 0}--\z{0x1FF} for \z{vsiselect}, even
though most or all of the space is inaccessible, permits a hypervisor
to emulate indirectly accessed registers in the implemented range,
including registers that may be standardized in the future at locations
\z{0x100}--\z{0x1FF}.
\end{commentary}

The locations of registers for external interrupts (numbers
\z{0x70}--\z{0xFF}) are accessible only when field VGEIN of \z{hstatus}
is the number of an implemented guest external interrupt, not zero.
If VGEIN is not the number of an implemented guest external interrupt
(including the case when no IMSIC is implemented), then all non-custom
values of \z{vsiselect} designate an inaccessible register.

Along the same lines, when \z{hstatus}.VGEIN is not the number of
an implemented guest external interrupt, attempts from \mbox{M-mode}
or \mbox{HS-mode} to access CSRs \z{vsseteipnum}, \z{vsclreipnum},
\z{vsseteienum}, \z{vsclreienum}, or \z{vstopei} raise an illegal instruction
exception, and attempts from \mbox{VS-mode} to access \z{sseteipnum},
\z{sclreipnum}, \z{sseteienum}, \z{sclreienum}, or \z{stopei} raise a
virtual instruction exception.

Aside from the high-half CSR \z{hidelegh}, the hypervisor CSRs added
by the Advanced Interrupt Architecture (\z{hvien}, \z{hvictl},
\z{hviprio1}, and \z{hviprio2}, plus the high-half hypervisor CSRs for
RV32) augment \z{hvip} for injecting interrupts into VS level.
The use of these registers is covered in Chapter~\ref{ch:VSLevel} on
interrupts for virtual machines.


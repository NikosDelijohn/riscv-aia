
%=======================================================================
\chapter{Interprocessor Interrupts (IPIs)}
\label{ch:IPIs}

\textbf{%
Warning!
This draft specification is likely to change before being accepted as
standard by the {\RISCV} International Association.%
}
\bigskip

By default, unless a platform has a different mechanism for
interprocessor interrupts (IPIs), the {\RISCV} Privileged Architecture
specifies that a machine with multiple harts must provide for each hart
an implementation-defined memory address that can be written to signal
a machine-level \emph{software interrupt} (major code~3) at that hart.
IPIs at machine level can thus be sent to any hart as machine-level
software interrupts.

\begin{commentary}
A {\RISCV} software interrupt acts only as a minimal ``doorbell''
signal.
Software at the receiving hart is responsible for recognizing an
incoming software interrupt as an IPI and decoding its purpose further,
usually making use of additional data stored by the sender in ordinary
memory.
\end{commentary}

The same kind of mechanism (but with a different set of addresses) may
or may not exist for signalling supervisor-level software interrupts
(major code~1) at remote harts as well.
If not directly supported in this way, then a supervisor-level software
interrupt is typically sent to another hart through an environment call
from supervisor mode to machine mode.
An operating system running in \mbox{S-mode} thus invokes a specific
SBI function for delivering a software interrupt to another hart,
causing machine-level software at the originating hart to send a
machine-level IPI to the destination hart, where software then sets the
supervisor-level software interrupt-pending bit (SSIP) in CSR \z{mip}.

When harts have IMSICs, instead of using the Privileged Architecture's
mechanism for signalling software interrupts at remote harts, an IPI
can be sent to a hart by writing to the destination hart's IMSIC, the
same as a regular message-signalled interrupt (MSI)\@.
In that case, an incoming IPI appears at the destination hart as an
\emph{external interrupt} routed through the IMSIC, rather than as a
software interrupt as before.
However, so long as the same software (e.g.\@ an operating system or
machine monitor) is in control at both endpoints of an IPI, source
and destination, there should be no reason for a destination hart
to misinterpret the purpose of an incoming external interrupt that
represents an IPI.

If harts do not have IMSICs, then the method specified by the {\RISCV}
Privileged Architecture is assumed to be used for IPIs, with software
interrupts signalled at destination harts.
On the other hand, when harts have IMSICs, the machinery for triggering
software interrupts at remote harts is redundant with the capabilities
of the IMSICs, so it is downgraded from a requirement to an option,
useful perhaps only to maximize software compatibility across a range
of {\RISCV} systems, with and without IMSICs.
If a machine implements IMSICs and not the earlier software-interrupt
mechanism, then the bits of CSRs \z{mip} and \z{mie} for machine-level
software interrupts, MSIP and MSIE, are hardwired to zero in harts.

\begin{commentary}
If a machine implements IMSICs but not the software-interrupt
mechanism, the latter can still be fully emulated at supervisor level
for \mbox{S-mode} or \mbox{VS-mode}, by trapping on writes to the
special memory addresses that should signal supervisor-level software
interrupts at remote harts.
On such a trap, software can send a higher-level IPI via IMSIC to the
destination hart, where the higher-level software then can set the
SSIP bit in \z{sip} at the intended privilege level, S or~VS.

Similarly, SBI functions for sending IPIs can easily continue to
be supported without clients being at all aware of a change in the
underlying hardware for delivering IPIs between harts.
\end{commentary}


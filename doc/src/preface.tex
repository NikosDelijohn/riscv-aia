
%=======================================================================
\chapter{Preface}

This document describes an Advanced Interrupt Architecture
for {\RISCV} systems.

No part of this document has yet been ratified
by the {\RISCV} International Association.
The table below shows the current status of each chapter,
and also indicates which chapters specify extensions to the
{\RISCV} ISA (instruction set architecture) and which are non-ISA.

{
\begin{table}[hbt]
\centering
\begin{tabular}{|l|c|c|}
\hline
Chapter                                                  & ISA? & Status \\
\hline
\hline
1.\ Introduction                                         & ---  & Frozen \\
2.\ Control and Status Registers (CSRs) Added to Harts   & Yes  & Frozen \\
3.\ Incoming MSI Controller (IMSIC)                      & Yes  & Frozen \\
4.\ Advanced Platform-Level Interrupt Controller (APLIC) & No   & Frozen \\
5.\ Interrupts for Machine and Supervisor Levels         & Yes  & Frozen \\
6.\ Interrupts for Virtual Machines (VS Level)           & Yes  & Frozen \\
7.\ Interprocessor Interrupts (IPIs)                     & No   & Frozen \\
8.\ IOMMU Support for MSIs to Virtual Machines           & No   & Frozen \\
\hline
\end{tabular}
\end{table}
}

An implementation adhering to the current document might not conform
to an eventual ratified Advanced Interrupt Architecture for {\RISCV}.
However, all chapters are \emph{frozen}, meaning they are not expected
to change significantly before being put up for ratification.

\section*{Changes for RC6 (Ratification Candidate 6)}

Resolved some inconsistencies in Chapter~\ref{ch:CSRs}
about when to raise a virtual instruction exception
versus an illegal instruction exception.

\section*{Changes for RC5}

Better aligned the rules for indirectly accessed registers with the
hypervisor extension and with forthcoming extension Smcsrind/Sscsrind.
In particular, when \z{vsiselect} has a reserved value, attempts
to access \z{sireg} from a virtual machine (VS or \mbox{VU-mode})
should preferably raise an illegal instruction exception
instead of a virtual instruction exception.

Added clarification about the term \emph{\mbox{IOMMU}}
used in Chapter~\ref{ch:IOMMU}.

Added clarification about MSI write replaced by MRIF update and
notice MSI sent after the update.

\section*{Changes for RC4}

For alignment with other forthcoming {\RISCV} ISA extensions,
the widths of the indirect-access CSRs, \z{miselect}, \z{mireg},
\z{siselect}, \z{sireg}, \z{vsiselect}, and \z{vsireg}, were
changed to all be the current XLEN rather than being tied to
their respective privilege levels (previously MXLEN for
\z{miselect} and \z{mireg}, SXLEN for \z{siselect} and \z{sireg},
and VSXLEN for \z{vsiselect} and \z{vsireg}).

Changed the description (but not the actual function)
of \emph{high-half} CSRs and their partner CSRs
to match the latest {\RISCV} Privileged ISA specification.
(An example of a high-half CSR is \z{miph},
and its partner here is \z{mip}.)

\section*{Changes for RC3}

Removed the still-draft Duo-PLIC chapter to a separate document.

Allocated major interrupts 35 and 43 for signaling RAS events
(Section~\ref{sec:majorIntrs}).

In Section~\ref{sec:virtIntrs-S} added the options
for bits 1 and 9 to be writable in CSR \z{mvien},
and specified the effects of setting each of these bits.

Upgraded Chapter~\ref{ch:IOMMU} (``IOMMU Support'')
to the \emph{frozen} state.

\section*{Changes for RC2}

The only normative changes to this document
between versions RC1 and RC2 were to clarify that
field IID of CSR \z{hvictl} must support all
unsigned integer values of the number of bits implemented
for that field, and that writes to \z{hvictl}
always set IID in the most straightforward way.

A comment was added to Chapter~\ref{ch:IPIs} warning about
the possible need for FENCE instructions when IPIs are
sent to other harts by writing MSIs to those harts' IMSICs.


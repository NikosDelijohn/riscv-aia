
%=======================================================================
\chapter{Interrupts for Machine and Supervisor Levels}
\label{ch:MSLevel}

\textbf{%
This chapter is \emph{frozen} and has already passed public review,
making a functional change at this stage extremely unlikely.%
}
\bigskip

The {\RISCV} Privileged Architecture defines several major identities
in the range 0--15 for interrupts at a hart, including machine-level
and supervisor-level external interrupts (numbers 11 and~9), machine-
and supervisor-level timer interrupts (7~and~5), and machine- and
supervisor-level software interrupts (3~and~1).
Beyond these major labels, the \emph{external} interrupts at each
privilege level are given secondary, minor identities by an external
interrupt controller such as an APLIC or IMSIC, distinguishing interrupts
from different devices or causes.
These minor identities for external interrupts were covered in Chapters
\ref{ch:IMSIC} and~\ref{ch:AdvPLIC} specifying the IMSIC and
APLIC components.

The Advanced Interrupt Architecture reserves another 24 major
interrupt identities for additional \emph{local interrupts}
that arise within or in close proximity to the hart, often for
reporting errors.
A mechanism is also defined that allows software to
selectively delegate both local and custom interrupts to the next lower
privilege level, or in some cases to inject entirely virtual interrupts
into a lower privilege level.

Lastly, an optional facility lets software assign priorities to major
interrupts (such as the timer and software interrupts, and any local
interrupts) such that they may mix with the priorities set for external
interrupts by a PLIC, APLIC, or IMSIC.

%-----------------------------------------------------------------------
\section{Defined major interrupts and default priorities}
\label{sec:majorIntrs}

Table~\ref{tab:majorIntrs} lists all the major interrupts
currently defined for {\RISCV} harts.
At this time, the only standard major interrupts are those
specified by the {\RISCV} Privileged Architecture.

\begin{table*}[h!]
\begin{center}
\begin{tabular}{|c|l|l|}
\hline
Default        &                         & \\
priority order & Major interrupt numbers & \multicolumn{1}{c|}{Description} \\
\hline
\hline
Highest & 11, 3, 7   & Machine interrupts:  external, software, timer \\
        & \ 9, 1, 5  & Supervisor interrupts:  external, software, timer \\
        & 12         & Supervisor guest external interrupt \\
        & 10, 2, 6   & VS interrupts:  external, software, timer \\
Lowest  & 13         & Counter overflow interrupt \\
\hline
\end{tabular}
\end{center}
\caption{%
The standard major interrupt codes, listed in default priority order.
Currently, all standard codes are defined by
the {\RISCV} Privileged Architecture.%
}
\label{tab:majorIntrs}
\end{table*}

\begin{table*}[h!]
\begin{center}
\begin{tabular}{|c|l|}
\hline
                          & \\
Major interrupt numbers   & \hspace{10mm}Category \\
\hline
\hline
\begin{tabular}{@{}c@{}}
  \ 0--12 \\
  13--15 \\
\end{tabular} &
    \begin{tabular}{@{}l@{}}
      Not local interrupts \\
      Local interrupts \\
    \end{tabular}
      $\left\}\begin{tabular}{l@{}}
        Assigned by the \\
        \quad Privileged Architecture \\
      \end{tabular}\right.$ \\
\hline
16--23                    & Local interrupts\\
24--31                    & \em Designated for custom use \\
32--47                    & Local interrupts \\
\mbox{\ $\geq \mbox{48}$} & \em Designated for custom use \\
\hline
\end{tabular}
\end{center}
\caption{Categorization of current and future major interrupts.}
\label{tab:majorIntrCategories}
\end{table*}

Of the major interrupts controlled by the Privileged Architecture
(numbers 0--15), this Advanced Interrupt Architecture (AIA) categorizes the
counter overflow interrupt (code~13) as a \emph{local interrupt}.
It is assumed furthermore that any future definitions for
reserved interrupt numbers 14 and 15 will also be local interrupts.
The AIA additionally reserves
major interrupt numbers 16--23 and 32--47 for
standard local interrupts that other {\RISCV} extensions may define.
The remaining major interrupts allocated to the Privileged
Architecture, numbers 0--12, are categorized as not local interrupts.
Taken altogether, Table~\ref{tab:majorIntrCategories}
summarizes the AIA's categorization
of all major interrupt identities.

\begin{commentary}
For the standard local interrupts not defined by the
{\RISCV} Privileged Architecture (numbers 16--23 and 32--47),
the current plan is to assign default priorities
in the order shown in this table:
\begin{center}
\begin{tabular}{|c|l|l|}
\hline
Highest & 47, 23, 46, 45, 22, 44, & \\
        & 43, 21, 42, 41, 20, 40  & \\
\cline{2-3}
        & 11, 3, 7   & Machine interrupts:  external, software, timer \\
        & \ 9, 1, 5  & Supervisor interrupts:  external, software, timer \\
        & 12         & Supervisor guest external interrupt \\
        & 10, 2, 6   & VS interrupts:  external, software, timer \\
        & 13         & Counter overflow interrupt \\
\cline{2-3}
        & 39, 19, 38, 37, 18, 36, & \\
Lowest  & 35, 17, 34, 33, 16, 32  & \\
\hline
\end{tabular}
\end{center}
Among interrupts 16--23, a higher interrupt number conveys
higher default priority, and likewise for interrupts 32--47.
These two groups are interleaved together in the complete order,
and the Privileged Architecture's standard interrupts, 0--15,
are inserted into the middle of the sequence.
This proposed default priority order is arranged so that
interrupts 0--31 can potentially be an adequate subset on their
own for \mbox{32-bit} {\RISCV} systems.

In actuality, future {\RISCV} extensions may or may not stick to
this plan for the default priority order of interrupts they define.
\end{commentary}

\begin{commentary}
In addition to the existing major interrupts of
Table~\ref{tab:majorIntrs}, the following
local interrupts are tentatively proposed,
listed in order of decreasing default priority:\nopagebreak
\begin{displayLinesTable}[l@{\quad}l]
23 & Bus or system error \\
45 & Per-core high-power or over-temperature event \\
43 & High-priority RAS event \\
\noalign{\smallskip}
35 & Low-priority RAS event \\
17 & Debug/trace interrupt \\
\end{displayLinesTable}
\noindent
These local interrupts are expected to be
specified by other {\RISCV} extensions.
Be aware, this list is not final and may change
as the relevant extensions are developed and ratified.
\emph{RAS} is an abbreviation for \emph{Reliability, Availability, and
Serviceability}.
\end{commentary}

\begin{commentary}
If a future version of the {\RISCV} Privileged Architecture
defines interrupt~0, the
Advanced Interrupt Architecture needs it to have a default priority
lower than certain external interrupts.
See Sections \ref{sec:mtopi} and~\ref{sec:stopi} on CSRs \z{mtopi} and
\z{stopi}.
\end{commentary}

Interrupt numbers 24--31 and 48 and higher
are all designated for custom use.
If a hart implements any custom interrupts, their
positions in the default priority order must be documented for the
hart.

\begin{commentary}
While many of the standard registers such as\/ \z{mip}/\z{miph}
and\/ \z{mie}/\z{mieh} have space for major interrupts
only in the range 0--63, custom interrupts with numbers
64 and above are conceivable with added custom support.
CSRs\/ \z{mtopi} (Section~\ref{sec:mtopi})
and\/ \z{stopi} (Section~\ref{sec:stopi}) allow for
major interrupt numbers potentially as large as 4095.
\end{commentary}

When a hart supports the arbitrary configuration of interrupt
priorities by software (described in later sections), the default
priority order still remains relevant for breaking ties when two
interrupt sources are assigned the same priority number.

%-----------------------------------------------------------------------
\section{Interrupts at machine level}

For whichever standard local interrupts are
implemented, the corresponding bits in CSRs
\z{mip}/\z{miph} and \z{mie}/\z{mieh} must be writable, and the
corresponding bits in \z{mideleg}/\z{midelegh} (if those CSRs exist
because supervisor mode is implemented) must each either be writable or
be hardwired to zero.
An occurrence of a local interrupt event causes the interrupt-pending bit
in \z{mip}/\z{miph} to be set to one.
This bit then remains set until cleared by software.

As established by the {\RISCV} Privileged Architecture, an interrupt
traps to \mbox{M-mode} whenever all of the following are true:
(a)~either the current privilege mode is \mbox{M-mode} and
machine-level interrupts are enabled by the MIE bit of \z{mstatus}, or
the current privilege mode has less privilege than \mbox{M-mode};
(b)~matching bits in \z{mip}/\z{miph} and \z{mie}/\z{mieh} are both
one; and
(c)~if \z{mideleg} exists, the corresponding bit in
\z{mideleg}/\z{midelegh} is zero.

When multiple interrupt causes are ready to trigger simultaneously, the
interrupt taken first is determined by priority order, which may be the
default order specified in the previous section
(\ref{sec:majorIntrs}), or may be a modified order configured
by software.

%- - - - - - - - - - - - - - - - - - - - - - - - - - - - - - - - - - - -
\subsection{Configuring priorities of major interrupts at machine level}
\label{sec:intrPrios-M}

The machine-level priorities for major interrupts 0--63 may be
configured by a set of registers accessed through the \z{miselect} and
\z{mireg} CSRs introduced in Chapter~\ref{ch:CSRs}.
When MXLEN =~32, sixteen of these registers are defined, listed below
with their \z{miselect} addresses:
\begin{displayLinesTable}[c@{\quad}l]
\z{0x30} & \z{iprio0} \\
\z{0x31} & \z{iprio1} \\
\dots    & \ \dots \\
\z{0x3F} & \z{iprio15} \\
\end{displayLinesTable}
Each register controls the priorities of four interrupts, with one
\mbox{8-bit} byte per interrupt.
For a number~$k$ in the range 0--15, register \z{iprio}$k$
controls the priorities of interrupts $k\times\mbox{4}$ through
${k\times\mbox{4}+\mbox{3}}$, formatted as follows:
\begin{displayLinesTable}[l@{\quad}l]
bits 7:0   & Priority number for interrupt $k\times\mbox{4}$ \\
bits 15:8  & Priority number for interrupt $k\times\mbox{4}+\mbox{1}$ \\
bits 23:16 & Priority number for interrupt $k\times\mbox{4}+\mbox{2}$ \\
bits 31:24 & Priority number for interrupt $k\times\mbox{4}+\mbox{3}$ \\
\end{displayLinesTable}

When MXLEN =~64, only the even-numbered registers exist:\nopagebreak
\begin{displayLinesTable}[c@{\quad}l]
\z{0x30} & \z{iprio0} \\
\z{0x32} & \z{iprio2} \\
\dots    & \ \dots \\
\z{0x3E} & \z{iprio14} \\
\end{displayLinesTable}
Each register controls the priorities of eight interrupts.
For even~$k$ in the range 0--14, register \z{iprio}$k$
controls the priorities of interrupts $k\times\mbox{4}$ through
${k\times\mbox{4}+\mbox{7}}$, formatted as follows:
\begin{displayLinesTable}[l@{\quad}l]
bits 7:0   & Priority number for interrupt $k\times\mbox{4}$ \\
bits 15:8  & Priority number for interrupt $k\times\mbox{4}+\mbox{1}$ \\
bits 23:16 & Priority number for interrupt $k\times\mbox{4}+\mbox{2}$ \\
bits 31:24 & Priority number for interrupt $k\times\mbox{4}+\mbox{3}$ \\
bits 39:32 & Priority number for interrupt $k\times\mbox{4}+\mbox{4}$ \\
bits 47:40 & Priority number for interrupt $k\times\mbox{4}+\mbox{5}$ \\
bits 55:48 & Priority number for interrupt $k\times\mbox{4}+\mbox{6}$ \\
bits 63:56 & Priority number for interrupt $k\times\mbox{4}+\mbox{7}$ \\
\end{displayLinesTable}

When MXLEN =~64 and \z{miselect} is an odd value in the range
\z{0x31}--\z{0x3F}, attempting to access \z{mireg} raises an illegal
instruction exception.

The valid registers \z{iprio0}--\z{iprio15} are known collectively as
the \emph{\texttt{iprio} array} for machine level.

The width of priority numbers for external interrupts is
\emph{IPRIOLEN}.
This parameter is affected by the main external interrupt controller
for the hart, whether a PLIC, APLIC, or IMSIC.

For an APLIC, IPRIOLEN is in the range 1--8 as specified in
Chapter~\ref{ch:AdvPLIC} on the APLIC.

For an IMSIC, IPRIOLEN is 6, 7, or~8.
IPRIOLEN may be 6 only if the number of external interrupt identities
implemented by the IMSIC is~63.
IPRIOLEN may be 7 only if the number of external interrupt identities
implemented by the IMSIC is no more than 127.
IPRIOLEN may be 8 for any IMSIC, regardless of the number of external
interrupt identities implemented.

Each byte of a valid \z{iprio}$k$ register is either a read-only zero
or a {\WARL} unsigned integer field implementing exactly IPRIOLEN bits.
For a given interrupt number, if the corresponding bit in
\z{mie}/\z{mieh} is read-only zero, then the interrupt's priority
number in the \z{iprio} array must be read-only zero as well.
The priority number for a machine-level external interrupt (bits 31:24
of register \z{iprio2}) must also be read-only zero.
Aside from these two restrictions, implementations may freely choose
which priority number fields are settable and which are read-only
zeros.
If all bytes in the \z{iprio} array are read-only zeros, priorities
can be configured only for external interrupts, not for any other
interrupts.

\begin{commentary}
Platform standards may require that priorities be configurable for
certain interrupt causes.
\end{commentary}

The \z{iprio} array accessed via \z{miselect} and \z{mireg} affects the
prioritization of interrupts only when they trap to \mbox{M-mode}.
When an interrupt's priority number in the array is zero (either
read-only zero or set to zero), its priority is the default order from
Section~\ref{sec:majorIntrs}.
Setting an interrupt's priority number instead to a nonzero value~$p$
gives that interrupt nominally the same priority as a machine-level
external interrupt with priority number~$p$.
For a major interrupt that defaults to a higher priority than machine
external interrupts, setting its priority number to a nonzero value
\emph{lowers} its priority.
For a major interrupt that defaults to a lower priority than machine
external interrupts, setting its priority number to a nonzero value
\emph{raises} its priority.
When two interrupt causes have been assigned the same nominal priority,
ties are broken by the default priority order.
Table~\ref{tab:intrPrios-M} summarizes the effect of priority numbers
on interrupt priority.

\begin{table*}[h!]
\begin{center}
\begin{tabular}{|c|c|c|c|}
\hline
 & Interrupts with default & Machine external & Interrupts with default \\
 & priority above machine  & interrupts       & priority below machine \\
 & external interrupts     &                  & external interrupts \\
\hline
         & Priority number in & Priority number from & Priority number in \\
Priority & machine-level      & interrupt controller & machine-level \\
order    & \z{iprio} array    & (APLIC or IMSIC)     & \z{iprio} array \\
\hline
\hline
Highest  & 0                  &                      & \\
\cline{2-4}
         & 1                  & 1                    & 1 \\
         & 2                  & 2                    & 2 \\
         & $\cdots$           & $\cdots$             & $\cdots$ \\
         & 254                & 254                  & 254 \\
         & 255                & 255                  & 255 \\
\cline{2-4}
         &                    & 256 and above        & \\
         &                    & (IMSIC only)         & \\
\cline{2-4}
Lowest   &                    &                      & 0 \\
\hline
\end{tabular}
\end{center}
\caption{%
Effect of the machine-level \z{iprio} array on the priorities of
interrupts taken in \mbox{M-mode}.
For interrupts with the same priority number, the default order of
Section~\ref{sec:majorIntrs} prevails.%
}
\label{tab:intrPrios-M}
\end{table*}

\begin{commentary}
When a hart has an IMSIC supporting more than 255 minor identities
for external interrupts, the only non-default priorities that can be
configured for other interrupts are those corresponding to external
interrupt identities 1--255, not those of identities 256 or higher.
\end{commentary}

\begin{commentary}
Implementing the priority configurability of this section requires
that a {\RISCV} hart's external interrupt controller communicate
to the hart not only the existence of a pending-and-enabled
external interrupt but also the interrupt's priority number.
Typically this implies that the width of the connection for signaling
an external interrupt to the hart is not just a single wire as usual
but now $\mbox{IPRIOLEN} + 1$ wires.

It is expected that many systems will forego
priority configurability of major interrupts and simply
have the\/ \z{iprio} array be all read-only zeros.
Systems that need this priority configurability can try
to arrange for each hart's external interrupt controller
to be relatively close to the hart, by, for example, limiting
the system to at most a few small cores connected to an APLIC,
or alternatively by giving every hart its own IMSIC.
\end{commentary}

If supported, setting the priority number for supervisor-level external
interrupts (bits 15:8 of \z{iprio2}) to a nonzero value~$p$ has the
effect of giving the entire category of supervisor external interrupts
nominally the same priority as a machine external interrupt with
priority number~$p$.
But note that this applies only to the case when supervisor external
interrupts trap to \mbox{M-mode}.

(Because supervisor guest external interrupts and VS-level external
interrupts are required to be delegated to supervisor level when the
hypervisor extension is implemented, the machine-level priority numbers
for these interrupts are always ignored and should be read-only zeros.)

If the system has an original PLIC for backward
compatibility with older software, reset should initialize the
machine-level \z{iprio} array to all zeros.

%- - - - - - - - - - - - - - - - - - - - - - - - - - - - - - - - - - - -
\subsection{Machine top interrupt CSR (\zSafe{mtopi})}
\label{sec:mtopi}

Machine-level CSR \z{mtopi} is read-only with width MXLEN.
A read of \z{mtopi} returns information about the highest-priority
pending-and-enabled interrupt for machine level, in this
format:\nopagebreak
\begin{displayLinesTable}[l@{\quad}l]
bits 27:16 & IID \\
bits 7:0   & IPRIO \\
\end{displayLinesTable}
All other bits of \z{mtopi} are reserved and read as zeros.

The value of \z{mtopi} is zero unless there is an interrupt pending in
\z{mip}/\z{miph} and enabled in \z{mie}/\z{mieh} that is not delegated
to a lower privilege level.
When there is a pending-and-enabled major interrupt for machine
level, field IID (Interrupt Identity) is the major identity number of
the highest-priority interrupt, and field IPRIO indicates its priority.

If all bytes of the machine-level \z{iprio} array are read-only zeros,
a simplified implementation of field IPRIO is allowed in which its
value is always 1 whenever \z{mtopi} is not zero.

Otherwise, when \z{mtopi} is not zero, if the priority number for the reported
interrupt is in the range 1 to 255, IPRIO is simply that number.
If the interrupt's priority number is zero or greater than 255,
IPRIO is set to either 0 or 255 as follows:
\begin{itemize}

\item
If the interrupt's priority number is greater than 255, then
IPRIO is 255 (lowest priority).

\item
If the interrupt's priority number is zero and interrupt number IID has
a default priority higher than a machine external interrupt, then IPRIO
is~0 (highest priority).

\item
If the interrupt's priority number is zero and interrupt number IID has
a default priority lower than a machine external interrupt, then IPRIO
is 255 (lowest priority).

\end{itemize}

\begin{commentary}
To ensure that \z{mtopi} is never zero when an interrupt is pending
and enabled for machine level, if major interrupt~0 can trap to
\hbox{M-mode}, it must have a default priority lower than a machine
external interrupt.
\end{commentary}

The value of \z{mtopi} is not affected by
the global interrupt enable MIE in CSR \z{mstatus}.

The {\RISCV} Privileged Architecture ensures that, when the value
of \z{mtopi} is not zero, a trap is taken to \mbox{M-mode} for the
interrupt indicated by field IID if either the current privilege
mode is~M and \z{mstatus}.MIE is one, or the current
privilege mode has less privilege than \mbox{M-mode}.
The trap itself does not cause the value of \z{mtopi} to change.

The following pseudocode shows how a machine-level trap handler might
read \z{mtopi} to avoid redundant restoring and saving of processor
registers when an interrupt arrives during the handling of another trap
(either a synchronous exception or an earlier interrupt):
\begin{displayLinesTable}
save processor registers \\
\z{i = }read CSR \z{mcause} \\
\z{if (i >= 0) \LB} \\
\qquad handle synchronous exception \z{i} \\
\qquad restore \z{mstatus} if necessary \\
\z{\RB}\\
\z{if (mstatus.MPIE == 1) \LB} \\
\qquad loop \z{\LB} \\
\qquad\qquad \z{i = }read CSR \z{mtopi} \\
\qquad\qquad \z{if (i == 0)} exit loop \\
\qquad\qquad \z{i = i>>16} \\
\qquad\qquad call the interrupt handler for major interrupt \z{i} \\
\qquad \z{\RB} \\
\z{\RB} \\
restore processor registers \\
return from trap \\
\end{displayLinesTable}
(This example can be further optimized, but with an increase in
complexity.)

%-----------------------------------------------------------------------
\section{Interrupt filtering and virtual interrupts for supervisor level}
\label{sec:virtIntrs-S}

When supervisor mode is implemented, the Advanced Interrupt
Architecture adds a facility for software filtering
of interrupts and for virtual interrupts, making use of new
CSRs \z{mvien}/\z{mvienh} (Machine Virtual Interrupt Enables) and
\z{mvip}/\z{mviph} (Machine Virtual Interrupt-Pending bits).
\emph{Interrupt filtering} permits a supervisor-level
interrupt (SEI or SSI) or local or custom interrupt
to trap to \mbox{M-mode} and then be selectively delegated by
software to supervisor level, even while the corresponding bit in
\z{mideleg}/\z{midelegh} remains zero.
The same hardware may also, under the right circumstances, allow
machine level to assert \emph{virtual interrupts} to supervisor level
that have no connection to any real interrupt events.

Just as with CSRs \z{mip}/\z{miph}, \z{mie}/\z{mieh}, and
\z{mideleg}/\z{midelegh}, each bit of registers \z{mvien}/\z{mvienh}
and \z{mvip}/\z{mviph} corresponds with an interrupt number in the
range 0--63.
When a bit in \z{mideleg}/\z{midelegh} is zero and the matching
bit in \z{mvien}/\z{mvienh} is one, then the same bit position
in \z{sip}/\z{siph} is an alias for the corresponding bit in
\z{mvip}/\z{mviph}.
A bit in \z{sip}/\z{siph} is read-only zero when the corresponding bits
in \z{mideleg}/\z{midelegh} and \z{mvien}/\z{mvienh} are both zero.
The combined effects of \z{mideleg}/\z{midelegh} and
\z{mvien}/\z{mvienh} on \z{sip}/\z{siph} and \z{sie}/\z{sieh} are
summarized in Table~\ref{tab:intrFilteringForS}.

\begin{table*}[h!]
\begin{center}
\begin{tabular}{|c|c||c|c|}
\hline
\z{mideleg[}$n$\z{]} & \z{mvien[}$n$\z{]} &
    \z{sip[}$n$\z{]} & \z{sie[}$n$\z{]} \\
\hline
\hline
0 & 0  & Read-only 0                & Read-only 0 \\
0 & 1  & Alias of \z{mvip[}$n$\z{]} & Writable \\
1 & -- & Alias of \z{mip[}$n$\z{]}  & Alias of \z{mie[}$n$\z{]} \\
\hline
\end{tabular}
\end{center}
\caption{%
The effects of \z{mideleg} and \z{mvien} on \z{sip} and \z{sie} (except
for the hypervisor extension's VS-level interrupts, which appear in
\z{hip} and \z{hie} instead of \z{sip} and \z{sie}).
A bit in \z{mvien} can be set to~1 only
for major interrupts 1, 9, and 13--63.
For interrupts 0--12,
aliases in \z{sip} may be read-only, as specified by the
{\RISCV} Privileged Architecture.%
}
\label{tab:intrFilteringForS}
\end{table*}

\begin{commentary}
The name of CSR \z{mvien} is not ``\/\z{mvie}'' because the function of
this register is more analogous to \z{mcounteren} than to \z{mie}.
The bits of\/ \z{mvien} control whether the virtual interrupt-pending
bits in register \z{mvip} are active and visible at supervisor level.
This is different than how the usual interrupt-enable bits (such as
in \z{mie}) mask pending interrupts.
\end{commentary}

A bit in \z{sie}/\z{sieh} is writable if and only if the corresponding
bit is set in either \z{mideleg}/\z{midelegh} or \z{mvien}/\z{mvienh}.
When an interrupt is delegated by \z{mideleg}/\z{midelegh}, the
writable bit in \z{sie}/\z{sieh} is an alias of the corresponding bit
in \z{mie}/\z{mieh};
else it is an independent writable bit.
As usual, bits that are not writable in \z{sie}/\z{sieh} must be
read-only zeros.

If a bit of \z{mideleg}/\z{midelegh} is zero and the corresponding bit
in \z{mvien}/\z{mvienh} is changed from zero to one, then the value
of the matching bit in \z{sie}/\z{sieh} becomes {\unspecified}.
Likewise, if a bit of \z{mvien}/\z{mvienh} is one and the corresponding
bit in \z{mideleg}/\z{midelegh} is changed from one to zero, the value
of the matching bit in \z{sie}/\z{sieh} again becomes {\unspecified}.

For interrupt numbers 13--63, implementations may freely choose which
bits of \z{mvien}/\z{mvienh} are writable and which bits are read-only
zero or one.
If such a bit in \z{mvien}/\z{mvienh} is read-only zero (preventing the
virtual interrupt from being enabled), the same bit should be read-only
zero in \z{mvip}/\z{mviph}.
All other bits for interrupts 13--63 must be writable in
\z{mvip}/\z{mviph}.

\begin{commentary}
Platform standards or other extensions may require that
bits of\/ \z{mvien} for certain interrupt causes
be writable, or be read-only zero or one.
\end{commentary}

The bits of \z{mvien} for supervisor software interrupts (code~1)
and supervisor external interrupts (code~9)
are each either writable or read-only zero;
they cannot be read-only ones.
The other bits of \z{mvien} for interrupts 0--12
are reserved and must be read-only zeros.

It is strongly recommended that bit~9 of \z{mvien} be writable.
Furthermore, if bit~1 (SSIP) of \z{mip} can be set
automatically by an interrupt controller and
not just by explicit writes to \z{mip} or \z{sip},
it is strongly recommended that bit~1 of \z{mvien} also be writable.

When bit~1 of \z{mvien} is zero, bit~1 of \z{mvip}
is an alias of the same bit (SSIP) of \z{mip}.
But when bit~1 of \z{mvien} is one, bit~1 of \z{mvip}
is a separate writable bit independent of \z{mip}.SSIP.
When the value of bit~1 of \z{mvien} is changed from zero to one,
the value of bit~1 of \z{mvip} becomes {\unspecified}.

Bit~5 of \z{mvip} is an alias of the same bit (STIP)
in \z{mip} when that bit is writable in \z{mip}.
When STIP is not writable in \z{mip} (such as when
\z{menvcfg}.STCE =~1), bit~5 of \z{mvip} is read-only zero.

When bit~9 of \z{mvien} is zero, bit~9 of \z{mvip} is
an alias of the software-writable bit~9 of \z{mip} (SEIP).
But when bit~9 of \z{mvien} is one, bit~9 of \z{mvip}
is a writable bit independent of \z{mip}.SEIP\@.
Unlike for bit~1, changing the value of bit~9 of \z{mvien}
does not affect the value of bit~9 of \z{mvip}.

\begin{commentary}
When bit~9 of\/ \z{mvien} is zero, bit~9 of\/ \z{mvip}
makes the software-writable SEIP bit of\/ \z{mip}
directly accessible by itself.
\end{commentary}

Except for bits 1, 5, and~9 as specified above, the bits of \z{mvip}
in the range 12:0 are reserved and must be read-only zeros.

The value of bit~9 of \z{mvien} has some additional consequences
for supervisor external interrupts:
\begin{itemize}

\item
When bit~9 of \z{mvien} is zero, the software-writable SEIP bit
(bit~9 of \z{mvip}) interacts with reads and writes of \z{mip}
in the way specified by the {\RISCV} Privileged Architecture.
In particular, for most purposes, the value of bit~9 of \z{mvip}
is logically ORed into the readable value of \z{mip}.SEIP\@.
But when bit~9 of \z{mvien} is one, bit SEIP in \z{mip} is
read-only and does not include the value of bit~9 of \z{mvip}.
Rather, the value of \z{mip}.SEIP is simply
the supervisor external interrupt signal from
the hart's external interrupt controller (APLIC or IMSIC).

\item
If the hart has an IMSIC, then when bit~9 of \z{mvien} is one,
attempts from S-mode to explicitly access the supervisor-level
interrupt file raise an illegal instruction exception.
The exception is raised for attempts to access CSR \z{stopei},
or to access \z{sireg} when \z{siselect} has
a value in the range \z{0x70}--\z{0xFF}.
Accesses to guest interrupt files
(through \z{vstopei} or \z{vsiselect} + \z{vsireg}) are not affected.

\end{itemize}

When the hypervisor extension is implemented, if a bit is
zero in the same position in both \z{mideleg}/\z{midelegh}
and \z{mvien}/\z{mvienh}, then that bit is read-only zero in
\z{hideleg}/\z{hidelegh} (in addition to being read-only zero
in \z{sip}/\z{siph}, \z{sie}/\z{sieh}, \z{hip}, and \z{hie}).
But if a bit for one of interrupts 13--63 is a one in
either \z{mideleg}/\z{midelegh} or \z{mvien}/\z{mvienh},
then the same bit in \z{hideleg}/\z{hidelegh} may be writable
or may be read-only zero, depending on the implementation.
No bits in \z{hideleg}/\z{hidelegh} are ever read-only ones.
The {\RISCV} Privileged Architecture further
constrains bits 12:0 of \z{hideleg}.

When supervisor mode is implemented, the minimal required
implementation of \z{mvien}/\z{mvienh} and \z{mvip}/\z{mviph} has all
bits being read-only zeros except for \z{mvip}
bits 1 and~9, and sometimes bit~5, each of which is
an alias of an existing writable bit in \z{mip}.
(Although, as noted, it is strongly recommended
that bit 9 of \z{mvien} also be writable.)
When supervisor mode is not implemented, registers \z{mvien}/\z{mvienh}
and \z{mvip}/\z{mviph} do not exist.


%-----------------------------------------------------------------------
\section{Interrupts at supervisor level}
\label{sec:intrs-S}

If a standard local interrupt
becomes pending (=~1) in \z{sip}/\z{siph}, the bit in \z{sip}/\z{siph}
is writable and will remain set until cleared by software.

Just as for machine level, the taking of interrupt traps at supervisor
level remains essentially the same as specified by the {\RISCV}
Privileged Architecture.
An interrupt traps into \mbox{S-mode} (or \mbox{HS-mode}) whenever all
of the following are true:
(a)~either the current privilege mode is \mbox{S-mode} and
supervisor-level interrupts are enabled by the SIE bit of \z{sstatus},
or the current privilege mode has less privilege than \mbox{S-mode};
(b)~matching bits in \z{sip}/\z{siph} and \z{sie}/\z{sieh} are both
one, or, if the hypervisor extension is implemented, matching bits in
\z{hip} and \z{hie} are both one; and
(c)~if the hypervisor extension is implemented, the corresponding bit
in \z{hideleg}/\z{hidelegh} is zero.

%- - - - - - - - - - - - - - - - - - - - - - - - - - - - - - - - - - - -
\subsection{Configuring priorities of major interrupts at supervisor level}
\label{sec:intrPrios-S}

Supervisor-level priorities for major interrupts 0--63
are optionally configurable
in an array of supervisor-level \z{iprio}$k$ registers accessed through
\z{siselect} and \z{sireg}.
This array has the same structure when SXLEN = 32 or 64 as does the
machine-level \z{iprio} array when MXLEN = 32 or 64, respectively.
To summarize, when SXLEN =~32, there are sixteen \mbox{32-bit}
registers with these \z{siselect} addresses:
\begin{displayLinesTable}[c@{\quad}l]
\z{0x30} & \z{iprio0} \\
\z{0x31} & \z{iprio1} \\
\dots    & \ \dots \\
\z{0x3F} & \z{iprio15} \\
\end{displayLinesTable}
Each register controls the priorities of four interrupts, one
\mbox{8-bit} byte per interrupt.
When SXLEN =~64, only the even-numbered registers exist:
\begin{displayLinesTable}[c@{\quad}l]
\z{0x30} & \z{iprio0} \\
\z{0x32} & \z{iprio2} \\
\dots    & \ \dots \\
\z{0x3E} & \z{iprio14} \\
\end{displayLinesTable}
Each register controls the priorities of eight interrupts.
If SXLEN = 64 and \z{siselect} is an odd value in the range
\z{0x31}--\z{0x3F}, attempting to access \z{sireg} raises an illegal
instruction exception.

The valid registers \z{iprio0}--\z{iprio15} are known collectively as
the \emph{\texttt{iprio} array} for supervisor level.
Each byte of a valid \z{iprio}$k$ register is either a read-only zero
or a {\WARL} unsigned integer field implementing exactly IPRIOLEN bits.

For a given interrupt number, if the corresponding bit in
\z{sie}/\z{sieh} is read-only zero, then the interrupt's priority
number in the supervisor-level \z{iprio} array must be read-only zero
as well.
The priority number for a supervisor-level external interrupt
(bits 15:8 of \z{iprio2}) must also be read-only zero.
Aside from these two restrictions, implementations may freely choose
which priority number fields are settable and which are read-only
zeros.

\begin{commentary}
As always, platform standards may require that priorities be
configurable for certain interrupt causes.
\end{commentary}

\begin{commentary}
It is expected that many higher-end systems will not support
the ability to configure the priorities of major interrupts
at supervisor level as described in this section.
Linux in particular is not designed to
take advantage of such facilities if provided.
The\/ \z{iprio} array must be accessible
but may simply be all read-only zeros.
\end{commentary}

The supervisor-level \z{iprio} array accessed via \z{siselect} and
\z{sireg} affects the prioritization of interrupts only when they trap
to \mbox{S-mode}.
When an interrupt's priority number in the array is zero (either
read-only zero or set to zero), its priority is the default order from
Section~\ref{sec:majorIntrs}.
Setting an interrupt's priority number instead to a nonzero value~$p$
gives that interrupt nominally the same priority as a supervisor-level
external interrupt with priority number~$p$.
For an interrupt that defaults to a higher priority than supervisor
external interrupts, setting its priority number to a nonzero value
lowers its priority.
For an interrupt that defaults to a lower priority than supervisor
external interrupts, setting its priority number to a nonzero value
raises its priority.
When two interrupt causes have been assigned the same nominal priority,
ties are broken by the default priority order.
Table~\ref{tab:intrPrios-S} summarizes the effect of priority numbers
on interrupt priority.

\begin{table*}[h!]
\begin{center}
\begin{tabular}{|c|c|c|c|}
\hline
 & Interrupts with default  & Supervisor external & Interrupts with default \\
 & priority above supervisor & interrupts       & priority below supervisor \\
 & external interrupts       &                  & external interrupts \\
\hline
         & Priority number in & Priority number from & Priority number in \\
Priority & supervisor-level   & interrupt controller & supervisor-level \\
order    & \z{iprio} array    & (APLIC or IMSIC)     & \z{iprio} array \\
\hline
\hline
Highest  & 0                  &                      & \\
\cline{2-4}
         & 1                  & 1                    & 1 \\
         & 2                  & 2                    & 2 \\
         & $\cdots$           & $\cdots$             & $\cdots$ \\
         & 254                & 254                  & 254 \\
         & 255                & 255                  & 255 \\
\cline{2-4}
         &                    & 256 and above        & \\
         &                    & (IMSIC only)         & \\
\cline{2-4}
Lowest   &                    &                      & 0 \\
\hline
\end{tabular}
\end{center}
\caption{%
Effect of the supervisor-level \z{iprio} array on the priorities of
interrupts taken in \mbox{S-mode}.
For interrupts with the same priority number, the default order of
Section~\ref{sec:majorIntrs} prevails.%
}
\label{tab:intrPrios-S}
\end{table*}

If supported, setting the priority number for VS-level external
interrupts (bits 23:16 of \z{iprio2}) to a nonzero value~$p$ has
the effect of giving the entire category of VS external interrupts
nominally the same priority as a supervisor external interrupt with
priority number~$p$, when VS external interrupts trap to \mbox{S-mode}.

If the system has an original PLIC for backward
compatibility with older software, reset should initialize the
supervisor-level \z{iprio} array to all zeros.

%- - - - - - - - - - - - - - - - - - - - - - - - - - - - - - - - - - - -
\subsection{Supervisor top interrupt CSR (\zSafe{stopi})}
\label{sec:stopi}

Supervisor-level CSR \z{stopi} is read-only with width SXLEN.
A read of \z{stopi} returns information about the highest-priority
pending-and-enabled interrupt for supervisor level, in this
format:\nopagebreak
\begin{displayLinesTable}[l@{\quad}l]
bits 27:16 & IID \\
bits 7:0   & IPRIO \\
\end{displayLinesTable}
All other bits of \z{stopi} are reserved and read as zeros.

The value of \z{stopi} is zero unless:
(a)~there is an interrupt that is both pending in \z{sip}/\z{siph}
and enabled in \z{sie}/\z{sieh}, or, if the hypervisor extension is
implemented, both pending in \z{hip} and enabled in \z{hie}; and
(b)~the interrupt is not delegated to a lower privilege level
(by \z{hideleg}, if the hypervisor extension is implemented).
When there is a pending-and-enabled major interrupt for supervisor
level, field IID is the major identity number of the highest-priority
interrupt, and field IPRIO indicates its priority.

If all bytes of the supervisor-level \z{iprio} array are read-only
zeros, a simplified implementation of field IPRIO is allowed in which
its value is always 1 whenever \z{stopi} is not zero.

Otherwise, when \z{stopi} is not zero, if the priority number for the reported
interrupt is in the range 1 to 255, IPRIO is simply that number.
If the interrupt's priority number is zero or greater than 255,
IPRIO is set to either 0 or 255 as follows:
\begin{itemize}

\item
If the interrupt's priority number is greater than 255, then
IPRIO is 255 (lowest priority).

\item
If the interrupt's priority number is zero and interrupt number IID has
a default priority higher than a supervisor external interrupt, then
IPRIO is~0 (highest priority).

\item
If the interrupt's priority number is zero and interrupt number IID
has a default priority lower than a supervisor external interrupt, then
IPRIO is 255 (lowest priority).

\end{itemize}

\begin{commentary}
To ensure that \z{stopi} is never zero when an interrupt is pending
and enabled for supervisor level, if major interrupt~0 can trap to
\hbox{S-mode}, it must have a default priority lower than a supervisor
external interrupt.
\end{commentary}

The value of \z{stopi} is not affected by
the global interrupt enable SIE in CSR \z{sstatus}.

The {\RISCV} Privileged Architecture ensures that, when the value
of \z{stopi} is not zero, a trap is taken to \mbox{S-mode} for the
interrupt indicated by field IID if either the current privilege
mode is~S and \z{sstatus}.SIE is one, or the current
privilege mode has less privilege than \mbox{S-mode}.
The trap itself does not cause the value of \z{stopi} to change.

The following pseudocode shows how a supervisor-level trap handler
might read \z{stopi} to avoid redundant restoring and saving of
processor registers when an interrupt arrives during the handling of
another trap (either a synchronous exception or an earlier interrupt):
\begin{displayLinesTable}
save processor registers \\
\z{i = }read CSR \z{scause} \\
\z{if (i >= 0) \LB} \\
\qquad handle synchronous exception \z{i} \\
\qquad restore \z{sstatus} if necessary \\
\z{\RB}\\
\z{if (sstatus.SPIE == 1) \LB} \\
\qquad loop \z{\LB} \\
\qquad\qquad \z{i = }read CSR \z{stopi} \\
\qquad\qquad \z{if (i == 0)} exit loop \\
\qquad\qquad \z{i = i>>16} \\
\qquad\qquad call the interrupt handler for major interrupt \z{i} \\
\qquad \z{\RB} \\
\z{\RB} \\
restore processor registers \\
return from trap \\
\end{displayLinesTable}
(This example can be further optimized, but with an increase in
complexity.)

%-----------------------------------------------------------------------
\section{WFI (Wait for Interrupt) instruction}

The {\RISCV} Privileged Architecture specifies that
instruction WFI (Wait for Interrupt) may suspend execution
at a hart until an interrupt is pending for the hart.
The Advanced Interrupt Architecture (AIA) redefines
when execution must resume following a WFI.

According to the {\RISCV} Privileged Architecture, instruction
execution must resume from a WFI whenever any interrupt
is both pending and enabled in CSRs \z{mip} and \z{mie},
ignoring any delegation indicated by \z{mideleg}.
With the AIA, this succinct rule is no longer appropriate,
due to the mechanisms the AIA adds for virtual interrupts.
Instead, execution must resume from a WFI whenever
an interrupt is pending at any privilege level
(regardless of whether the interrupt privilege level is
higher or lower than the hart's current privilege mode).

An interrupt is pending at machine level
if register \z{mtopi} is not zero.
If \mbox{S-mode} is implemented, an interrupt is
pending at supervisor level if \z{stopi} is not zero.
And if the hypervisor extension is implemented, an interrupt is pending
at VS level if \z{vstopi} (Section~\ref{sec:vstopi}) is not zero.

\begin{commentary}
The AIA's rule for WFI gives the same behavior as the Privileged
Architecture's rule when\/ \z{mvien} =~0 and, if the hypervisor
extension is implemented, also \z{hvien} =~0 and\/ \z{hvictl}.VTI =~0,
thus disabling all virtual interrupts not visible in\/ \z{mip}.
(The AIA's hypervisor registers are covered in the next chapter,
``Interrupts for Virtual Machines (VS Level)''.)
\end{commentary}


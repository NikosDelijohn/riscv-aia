
%=======================================================================
\chapter{Interrupts for Virtual Machines (VS Level)}
\label{ch:VSLevel}

\textbf{%
Warning!
This draft specification is likely to change before being accepted as
standard by the {\RISCV} International Association.%
}
\bigskip

When the hypervisor extension is implemented, a hart's set of possible
privilege modes includes the \emph{virtual supervisor} (VS) and
\emph{virtual user} (VU) modes for hosting virtual harts.
The Advanced Interrupt Architecture adds to the hypervisor extension
new interrupt facilities aligned with those described earlier for
supervisor-level interrupts.

As introduced in Chapter~\ref{ch:CSRs}, several hypervisor and VS CSRs
are added:  \z{hvien}, \z{hvictl}, \z{hviprio1}, \z{hviprio2},
\z{vsiselect}, \z{vsireg}, \z{vstopi}, \z{vsseteipnum},
\z{vsclreipnum}, \z{vsseteienum}, \z{vsclreienum}, and \z{vstopei}.
For RV32, the following ``high-half'' CSR partners are also added:
\z{hidelegh}, \z{hvienh}, \z{hviph}, \z{hviprio1h}, \z{hviprio2h},
\z{vsiph}, and \z{vsieh}.
As always, when executing in \mbox{VS-mode} or \mbox{VU-mode}, the
VS CSRs substitute for the corresponding supervisor CSRs.

To give software that runs in a virtual machine the appearance of
executing on a real machine that implements the Advanced Interrupt
Architecture at supervisor level, responsibility is shared between
hypervisor software and the hardware facilities described in this
chapter.
While some behaviors can be handled directly by hardware, others
require significant emulation by the hypervisor, sometimes with
hardware assistance.

%-----------------------------------------------------------------------
\section{VS-level external interrupts with a guest interrupt file}

When a hart implements the hypervisor extension, it is recommended that
the hart also have an IMSIC with guest interrupt files.
Assuming guest interrupt files are available, each can be assigned
to a virtual hart at the physical hart to act as the supervisor-level
interrupt file for that virtual hart.
If there are $N$~guest interrupt files, then $N$~virtual harts at that
physical hart may each have a physical guest interrupt file to serve as
its (virtual) supervisor-level interrupt file.
The guest interrupt file for the current virtual hart is always
indicated by the VGEIN field of CSR \z{hstatus}.
When VGEIN is not the valid number of a guest interrupt file, the
current virtual hart has no guest interrupt file to act as its
supervisor-level interrupt file.

When \z{hstatus}.VGEIN is the valid number of a guest interrupt
file, values of \z{vsiselect} in the range \z{0x70}--\z{0xFF} select
registers of this guest interrupt file, just as values of \z{siselect}
in the same range select registers of the IMSIC's true supervisor-level
interrupt file.
The registers of an interrupt file that are accessed indirectly through
\z{vsiselect} and \z{vsireg} are documented in Chapter~\ref{ch:IMSIC}
on the IMSIC, along with the effects of accessing CSRs \z{vsseteipnum},
\z{vsclreipnum}, \z{vsseteienum}, \z{vsclreienum}, and \z{vstopei}.
Because all IMSIC interrupt files act identically, the guest interrupt
file that a virtual hart accesses through CSRs \z{siselect}, \z{sireg},
\z{sseteipnum}, \z{sclreipnum}, \z{sseteienum}, \z{sclreienum},
and \z{stopei} is indistinguishable from a true supervisor-level
interrupt file as seen from \mbox{S-mode} (or \mbox{HS-mode}).

In addition to an IMSIC at each hart, a virtual machine may also need
to see a PLIC or APLIC.
However, unlike an IMSIC's ability to provide physical guest interrupt
files for virtual harts, a PLIC or APLIC must be emulated for a virtual machine
by the hypervisor.

\begin{commentary}
The Advanced Interrupt Architecture does not currently include hardware
assistance for virtualizing an APLIC.
For small numbers of harts, such hardware would be substantially larger
than that required to implement guest interrupt files for an IMSIC.
It is assumed that most high-performance I/O can be done through
devices that can send MSIs directly to guest interrupt files (such as
devices attached through a PCI Express interconnect).
For the types of devices whose interrupts must go through a (virtual)
APLIC, the overhead cost of emulating the APLIC is expected to be less
significant.
\end{commentary}

When a virtual hart appears to have an IMSIC because a guest interrupt
file is assigned to it, all external interrupts, real or emulated,
destined for the virtual hart must go through this perceived IMSIC.
A hypervisor can easily inject an emulated external interrupt into the
guest interrupt file selected by \z{hstatus}.VGEIN simply by writing to
CSR \z{vsseteipnum}.
When a virtual hart has a guest interrupt file, there is no reason
for a hypervisor to set bit VSEIP in CSR \z{hvip}, as this method
of asserting a VS-level external interrupt is more trouble than just
writing to \z{vsseteipnum} to set a specific interrupt-pending bit in
the guest interrupt file.

In the special case that an emulated APLIC for a virtual
machine has a wired interrupt source that equates to an actual
interrupt source of a real APLIC, if software running in this
virtual machine configures its virtual APLIC to forward interrupts from
that source as MSIs to a specific virtual hart, the hypervisor can
configure the real APLIC to forward the actual interrupts directly as
MSIs to the virtual hart's guest interrupt file.
In this way, although the hypervisor must trap and emulate the virtual
machine's memory accesses that configure the forwarding of interrupts
at the virtual APLIC, the interrupts themselves can be converted
automatically into real MSIs for the guest interrupt file, without the
hypervisor being invoked for each arriving interrupt.

%- - - - - - - - - - - - - - - - - - - - - - - - - - - - - - - - - - - -
\subsection{Direct control of a device by a guest OS}

To ensure proper support for interrupts, two conditions must be met
before a hypervisor may allow a guest OS running in a virtual machine
to directly control a physical device that sends MSIs:
First, each virtual hart must have a guest interrupt file assigned to
it, giving each its own apparent IMSIC within the virtual machine.
Second, interrupts from the device must be signaled by wire through
an APLIC that can translate these interrupts into MSIs, or the
system must have an I/O MMU that can translate the addresses of MSI
memory writes made by the device itself.

If a guest OS directly controls a device capable of sending MSIs, it
will naturally configure MSIs at the device with the guest physical
addresses the OS sees for the IMSICs of its virtual harts, not knowing
that these addresses are only virtual.
When the device performs a memory write for an MSI, the destination
address of this write must be translated by the I/O MMU from the
guest physical address assigned by the guest OS into the true physical
address of the target guest interrupt file, using a translation table
supplied by the hypervisor.

By design, the translation an I/O MMU must do for device MSIs is
fundamentally no different than the address translation the I/O MMU
already must perform for other memory accesses from the same device,
converting guest physical addresses into true physical addresses.
Because each virtual hart is assigned a dedicated, physical guest
interrupt file that is indistinguishable from a true supervisor-level
interrupt file, no translation is needed for the data of an MSI write,
which specifies the interrupt's identity number in the target interrupt
file.

%- - - - - - - - - - - - - - - - - - - - - - - - - - - - - - - - - - - -
\subsection{Migrating a virtual hart to a different guest interrupt file}
\label{sec:virtHartMigration}

When it is necessary to move a virtual hart from one physical hart to
another, if the virtual hart uses a guest interrupt file, the specific
guest interrupt file assigned to it must change from the one in use at
the old physical hart to a different one at the new physical hart.
Because each guest interrupt file is physically tied to a single
physical hart, a virtual hart cannot bring its guest interrupt file
with it when it moves.

The process of migrating a virtual hart from one guest interrupt file
to another is more complex than moving most other state held by the
virtual hart.
After the destination guest interrupt file has been chosen at the new
physical hart, the following steps are recommended:
\begin{enumerate}

\item
At the old interrupt file, save to memory the values of registers
\z{eidelivery} and \z{eithreshold}, and set \z{eidelivery} =~0.

\item
At the new interrupt file, set \z{eidelivery} =~0, and zero all
implemented interrupt-pending bits (the \z{eip} array).

\item
Modify the relevant translation tables at all I/O MMUs so that MSIs for
this virtual interrupt file are now sent to the new physical interrupt
file.
Likewise, if any interrupts at an APLIC are forwarded by MSIs to the old
interrupt file, reconfigure the APLIC to send them to the new interrupt
file.
As needed, synchronize with all I/O MMUs and APLICs to ensure that no
straggler MSIs will arrive at the old interrupt file after this step.
Synchronizing with an APLIC can be accomplished using the
algorithm of Section~\ref{sec:AdvPLIC-MSISync}.

\item
At the old interrupt file, dump to memory all implemented
interrupt-pending and interrupt-enable bits (the \z{eip} and \z{eie}
arrays).
After this step is done, the old interrupt file is no longer in use.

\item
At the new interrupt file, logically OR the interrupt-pending bits that
were saved in step~4 into the new interrupt file, using instruction
CSRS to write to the \z{eip} array.
Also, load the interrupt-enable bits that were saved in step~4 into the
\z{eie} array.

\item
At the new interrupt file, load registers \z{eithreshold} and
\z{eidelivery} with the values that were saved in step~1.

\end{enumerate}

Resuming execution of the virtual hart at the new physical hart is not
recommended until the entire interrupt file has been fully migrated.

\begin{commentary}
Resuming execution of the virtual hart before the interrupt file is
fully migrated could allow software running in the virtual machine to
see multiple MSIs arriving from a single device in an order that should
not happen.
While this would rarely matter in practice, it runs the risk of
wedging a device driver that depends (perhaps inadvertently) on a valid
ordering of events.
\end{commentary}

%-----------------------------------------------------------------------
\section{VS-level external interrupts without a guest interrupt file}

Although it is recommended that harts implementing the hypervisor
extension also have IMSICs with guest interrupt files, this is not a
requirement.
Even assuming guest interrupt files exist, it may happen that there
are more virtual harts at a physical hart than guest interrupt files,
leaving some virtual harts without one.
In either case, a hypervisor must emulate an external interrupt
controller for a virtual hart without the benefit of a guest interrupt
file allocated to the virtual hart.

When emulating an external interrupt controller for a virtual hart,
if configurable interrupt priority is not supported for the virtual
hart other than for external interrupts, then external interrupts may
be asserted to VS level simply by setting bit VSEIP in \z{hvip}, as
defined by the {\RISCV} Privileged Architecture.
However, to emulate both an external interrupt controller and priority
configurability for non-external interrupts, a hypervisor must make use
of CSR \z{hvictl} (Hypervisor Virtual Interrupt Control), described
later in the next section.

%-----------------------------------------------------------------------
\section{Interrupts at VS level}

%- - - - - - - - - - - - - - - - - - - - - - - - - - - - - - - - - - - -
\subsection{Configuring priorities of major interrupts at VS level}

Like for supervisor level, the Advanced Interrupt Architecture
optionally allows major VS-level interrupts to be configured by
software to intermix in priority with VS-level external interrupts.
As documented in Section~\ref{sec:intrs-S}, interrupt priorities
for supervisor level are configured by the \z{iprio} array accessed
indirectly through CSRs \z{siselect} and \z{sireg}.
The \z{siselect} addresses for the \z{iprio} array registers are
\z{0x30}--\z{0x3F}.

VS level has its own \z{vsiselect} and \z{vsireg}, but unlike
supervisor level, there are no registers at \z{vsiselect} addresses
\z{0x30}--\z{0x3F}.
When \z{vsiselect} has a value in the range \z{0x30}--\z{0x3F}, an
attempt from \mbox{VS-mode} to access \z{sireg} (really \z{vsireg})
causes a virtual instruction exception.
To give a virtual hart the illusion of an array of \z{iprio} registers
accessed through \z{siselect} and \z{sireg}, a hypervisor must
emulate the VS-level \z{iprio} array when accesses to \z{sireg} from
\mbox{VS-mode} cause virtual instruction traps.

Instead of a physical VS-level \z{iprio} array, a separate hardware
mechanism is provided for configuring the priorities of a subset
of interrupts for VS level, using hypervisor CSRs \z{hviprio1} and
\z{hviprio2}, plus, for RV32, their high-half partners, \z{hviprio1h}
and \z{hviprio2h}.
The subset of major interrupt numbers whose priority may be configured
in hardware are these:
\begin{displayLinesTable}[c@{\quad}l]
\ 1    & Supervisor software interrupt \\
\ 5    & Supervisor timer interrupt \\
 13    & Counter overflow interrupt \\
14--23 & \em Reserved for standard local interrupts \\
\end{displayLinesTable}
For interrupts directed to VS level, software-configurable priorities
are not supported in hardware for standard local interrupts
in the range 32--48.

\begin{commentary}
For custom interrupts, priority configurability may be
supported in hardware by custom CSRs, expanding upon
\z{hviprio1} and \z{hviprio2} for standard interrupts.
\end{commentary}

For RV32, registers \z{hviprio1}, \z{hviprio1h}, \z{hviprio2}, and
\z{hviprio2h} have these formats:

RV32 \z{hviprio1}:\nopagebreak
\begin{displayLinesTable}[l@{\quad}l]
bits 7:0   & \em Reserved for priority number for interrupt 0; reads as zero \\
bits 15:8  & Priority number for interrupt 1, supervisor software interrupt \\
bits 23:16 & \em Reserved for priority number for interrupt 4; reads as zero \\
bits 31:24 & Priority number for interrupt 5, supervisor timer interrupt \\
\end{displayLinesTable}
RV32 \z{hviprio1h}:\nopagebreak
\begin{displayLinesTable}[l@{\quad}l]
bits 7:0   & \em Reserved for priority number for interrupt 8; reads as zero \\
bits 15:8  & Priority number for interrupt 13, counter overflow interrupt \\
bits 23:16 & Priority number for interrupt 14 \\
bits 31:24 & Priority number for interrupt 15 \\
\end{displayLinesTable}
RV32 \z{hviprio2}:\nopagebreak
\begin{displayLinesTable}[l@{\quad}l]
bits 7:0   & Priority number for interrupt 16 \\
bits 15:8  & Priority number for interrupt 17 \\
bits 23:16 & Priority number for interrupt 18 \\
bits 31:24 & Priority number for interrupt 19 \\
\end{displayLinesTable}
RV32 \z{hviprio2h}:\nopagebreak
\begin{displayLinesTable}[l@{\quad}l]
bits 7:0   & Priority number for interrupt 20 \\
bits 15:8  & Priority number for interrupt 21 \\
bits 23:16 & Priority number for interrupt 22 \\
bits 31:24 & Priority number for interrupt 23 \\
\end{displayLinesTable}

For RV64, registers \z{hviprio1} and \z{hviprio2} are the \mbox{64-bit}
concatenations of the same RV32 registers with their respective
high-half partners:

RV64 \z{hviprio1}:\nopagebreak
\begin{displayLinesTable}[l@{\quad}l]
bits 7:0   & \em Reserved for priority number for interrupt 0; reads as zero \\
bits 15:8  & Priority number for interrupt 1, supervisor software interrupt \\
bits 23:16 & \em Reserved for priority number for interrupt 4; reads as zero \\
bits 31:24 & Priority number for interrupt 5, supervisor timer interrupt \\
bits 39:32 & \em Reserved for priority number for interrupt 8; reads as zero \\
bits 47:40 & Priority number for interrupt 13, counter overflow interrupt \\
bits 55:48 & Priority number for interrupt 14 \\
bits 63:56 & Priority number for interrupt 15 \\
\end{displayLinesTable}
RV64 \z{hviprio2}:\nopagebreak
\begin{displayLinesTable}[l@{\quad}l]
bits 7:0   & Priority number for interrupt 16 \\
bits 15:8  & Priority number for interrupt 17 \\
bits 23:16 & Priority number for interrupt 18 \\
bits 31:24 & Priority number for interrupt 19 \\
bits 39:32 & Priority number for interrupt 20 \\
bits 47:40 & Priority number for interrupt 21 \\
bits 55:48 & Priority number for interrupt 22 \\
bits 63:56 & Priority number for interrupt 23 \\
\end{displayLinesTable}

Each priority number in \z{hviprio1}/\z{hviprio1h} and
\z{hviprio2}/\z{hviprio2h} is a {\WARL} unsigned integer field that
is either read-only zero or implements a minimum of IPRIOLEN bits or
6~bits,
whichever is larger, and preferably all 8~bits.
Implementations may freely choose which priority number fields are
read-only zeros, but all other fields must implement the same number of
integer bits.
A minimal implementation of these CSRs has them all be read-only zeros.

A hypervisor can choose to employ registers \z{hviprio1}/\z{hviprio1h}
and \z{hviprio2}/\z{hviprio2h} when emulating the (virtual)
supervisor-level \z{iprio} array accessed indirectly through
\z{siselect} and \z{sireg} (really \z{vsiselect} and \z{vsireg}) for a
virtual hart.
For interrupts not in the subset supported by
\z{hviprio1}/\z{hviprio1h} and \z{hviprio2}/\z{hviprio2h}, the priority
number bytes in the emulated \z{iprio} array can be read-only zeros.

\begin{commentary}
Providing hardware support for configurable priority for only a subset
of major interrupts at VS level is a compromise.
The utility of being able to control interrupt priorities at
VS level is arguably illusory when all traps to \mbox{M-mode} and
\mbox{HS-mode}---both interrupts and synchronous exceptions---have
absolute priority, and when each virtual hart may also be competing for
resources against other virtual harts well beyond its control.
Nevertheless, priority configurability has been made possible for the
most likely subset of interrupts, while minimizing the number of added
CSRs that must be swapped on a virtual hart switch.

Major interrupts outside the priority-configurable subset can still
be directed to VS level, but their priority will simply be the default
order defined in Section~\ref{sec:majorIntrs}.
\end{commentary}

If a hypervisor really must emulate configurability of priority for
interrupts beyond the subset supported by \z{hviprio1}/\z{hviprio1h}
and \z{hviprio2}/\z{hviprio2h}, it can do so with extra effort
by setting bit VTI of CSR \z{hvictl}, described in the next
subsection.

%- - - - - - - - - - - - - - - - - - - - - - - - - - - - - - - - - - - -
\subsection{Virtual interrupts for VS level}

Assuming a virtual hart does not need configurable priority for
major interrupts beyond the subset supported in hardware by
\z{hviprio1}/\z{hviprio1h} and \z{hviprio2}/\z{hviprio2h}, a hypervisor
can assert interrupts to the virtual hart using CSRs \z{hvien}
(Hypervisor Virtual-Interrupt-Enable) and \z{hvip} (Hypervisor
Virtual-Interrupt-Pending bits), plus their high-half partners for RV32,
\z{hvienh} and \z{hviph}.
These CSRs affect interrupts for VS level much the same way that
\z{mvien}/\z{mvienh} and \z{mvip}/\z{mviph} do for supervisor level, as
explained in Section~\ref{sec:virtIntrs-S}.

Each bit of registers \z{hvien}/\z{hvienh} and \z{hvip}/\z{hviph}
corresponds with an interrupt number in the range 0--63.
Bits 12:0 of \z{hvien} are reserved and must be read-only zeros,
while bits 12:0 of \z{hvip} are defined by the {\RISCV} Privileged
Architecture.
Specifically, bits 10, 6, and 2 of \z{hvip} are writable bits that
correspond to VS-level external interrupts (VSEIP), VS-level timer
interrupts (VSTIP), and VS-level software interrupts (VSSIP),
respectively.

The following applies only to the CSR bits for interrupt numbers 13--63:
When a bit in \z{hideleg}/\z{hidelegh} is one, then the same bit
position in \z{vsip}/\z{vsiph} is an alias for the corresponding bit in
\z{sip}/\z{siph}.
Else, when a bit in \z{hideleg}/\z{hidelegh} is zero and the
matching bit in \z{hvien}/\z{hvienh} is one, the same bit position
in \z{vsip}/\z{vsiph} is an alias for the corresponding bit in
\z{hvip}/\z{hviph}.
A bit in \z{vsip}/\z{vsiph} is read-only zero when the corresponding
bits in \z{hideleg}/\z{hidelegh} and \z{hvien}/\z{hvienh} are both
zero.
The combined effects of \z{hideleg}/\z{hidelegh} and
\z{hvien}/\z{hvienh} on \z{vsip}/\z{vsiph} and \z{vsie}/\z{vsieh} are
summarized in Table~\ref{tab:intrFilteringForVS}.

\begin{table*}[h!]
\begin{center}
\begin{tabular}{|c|c||c|c|}
\hline
\z{hideleg[}$n$\z{]} & \z{hvien[}$n$\z{]} &
    \z{vsip[}$n$\z{]} & \z{vsie[}$n$\z{]} \\
\hline
\hline
0 & 0  & Read-only 0                & Read-only 0 \\
0 & 1  & Alias of \z{hvip[}$n$\z{]} & Writable \\
1 & -- & Alias of \z{sip[}$n$\z{]}  & Alias of \z{sie[}$n$\z{]} \\
\hline
\end{tabular}
\end{center}
\caption{%
The effects of \z{hideleg} and \z{hvien} on \z{vsip} and \z{vsie} for
major interrupts 13--63.%
}
\label{tab:intrFilteringForVS}
\end{table*}

For interrupt numbers 13--63, a bit in \z{vsie}/\z{vsieh} is
writable if and only if the corresponding bit is set in either
\z{hideleg}/\z{hidelegh} or \z{hvien}/\z{hvienh}.
When an interrupt is delegated by \z{hideleg}/\z{hidelegh}, the
writable bit in \z{vsie}/\z{vsieh} is an alias of the corresponding bit
in \z{sie}/\z{sieh};
else it is an independent writable bit.
The Privileged Architecture specifies when bits 12:0 of \z{vsie} are
aliases of bits in \z{hie}.
As usual, bits that are not writable in \z{vsie}/\z{vsieh} must be
read-only zeros.

If a bit of \z{hideleg}/\z{hidelegh} is zero and the corresponding bit 
in \z{hvien}/\z{hvienh} is changed from zero to one, then the value 
of the matching bit in \z{vsie}/\z{vsieh} becomes {\unspecified}.
Likewise, if a bit of \z{hvien}/\z{hvienh} is one and the corresponding 
bit in \z{hideleg}/\z{hidelegh} is changed from one to zero, the value
of the matching bit in \z{vsie}/\z{vsieh} again becomes {\unspecified}.

For interrupt numbers 13--63, implementations may freely choose which
bits of \z{hvien}/\z{hvienh} are writable and which bits are read-only
zero or one.
If such a bit in \z{hvien}/\z{hvienh} is read-only zero (preventing the
virtual interrupt from being enabled), the same bit should be read-only
zero in \z{hvip}/\z{hviph}.
All other bits for interrupts 13--63 must be writable in
\z{hvip}/\z{hviph}.

CSR \z{hvictl} (Hypervisor Virtual Interrupt Control) provides
further flexibility for injecting interrupts into VS level in
situations not fully supported by the facilities described thus far,
but only with more active involvement of the hypervisor.
A hypervisor must use \z{hvictl} for any of the following:
\begin{itemize}

\item
asserting for VS level a major interrupt not supported by
\z{hvien}/\z{hvienh} and \z{hvip}/\z{hviph};

\item
implementing configurability of priorities at VS level for major
interrupts beyond those supported by \z{hviprio1}/\z{hviprio1h} and
\z{hviprio2}/\z{hviprio2}; or

\item
emulating an external interrupt controller for a virtual hart without
the use of an IMSIC's guest interrupt file, while also supporting
configurable priorities both for external interrupts and for major
interrupts to the virtual hart.

\end{itemize}

The format of \z{hvictl} is:\nopagebreak
\begin{displayLinesTable}[l@{\quad}l]
bit 30     & VTI \\
bits 27:16 & IID (\WARL) \\
bit 8      & IPRIOM \\
bits 7:0   & IPRIO \\
\end{displayLinesTable}
All other bits of \z{hvictl} are reserved and read as zeros.

When bit VTI (Virtual Trap Interrupt control) =~1, attempts from
\mbox{VS-mode} to explicitly access CSRs \z{sip}, \z{sie}, or, for RV32
only, \z{siph} or \z{sieh}, cause a virtual instruction exception.
Furthermore, for any given CSR, if there is some circumstance in which
a write to the register may cause a bit of \z{vsip}/\z{vsiph} to change
from one to zero, excluding bit~9 for external interrupts (SEIP),
then when VTI =~1, a virtual instruction exception is raised
also for any attempt by the guest to write this register.
Both the value being written to the CSR and the value of
\z{vsip}/\z{vsiph} (before or after) are ignored for determining
whether to raise the exception.
(Hence a write would not actually need to change a bit of
\z{vsip}/\z{vsiph} from one to zero for the exception to be raised.)
In particular, if register \z{vstimecmp} is implemented
(from extension Sstc), then attempts from \mbox{VS-mode}
to write to \z{stimecmp} or (RV32 only) \z{stimecmph}
cause a virtual instruction exception when VTI~=~1.

\begin{commentary}
For the standard local interrupts
(major identities 13--23 and 32--47), and for
software interrupts (SSI), the corresponding interrupt-pending
bits in \z{vsip}/\z{vsiph} are defined as ``sticky,''
meaning a guest can clear them only by writing directly
to \z{sip}/\z{siph} (really \z{vsip}/\z{vsiph}).
Among the standard-defined interrupts, that leaves only
timer interrupts (STI), which can potentially be cleared
in \z{vsip} by writing a new value to \z{vstimecmp}.
\end{commentary}

All \z{hvictl} fields together can affect
the value of CSR \z{vstopi} (Virtual Supervisor Top Interrupt) and
therefore the interrupt identity reported in \z{vscause} when an
interrupt traps to \mbox{VS-mode}.
IID is a {\WARL} unsigned integer field with at least 6 implemented
bits, while IPRIO is always the full 8~bits.

%- - - - - - - - - - - - - - - - - - - - - - - - - - - - - - - - - - - -
\subsection{Virtual supervisor top interrupt CSR (\zSafe{vstopi})}

Read-only CSR \z{vstopi} is VSXLEN bits wide and has the same format as
\z{stopi}:\nopagebreak
\begin{displayLinesTable}[l@{\quad}l]
bits 27:16 & IID \\
bits 7:0   & IPRIO \\
\end{displayLinesTable}

\z{vstopi} returns information about the highest-priority interrupt for
VS level, found from among these candidates (prefixed by +~signs):
\begin{itemize}

\item
if \z{hstatus}.VGEIN is the valid number of a guest interrupt file,
bit~9 is one in both \z{vsip} and \z{vsie}, and \z{vstopei}
is not zero:
\begin{tightList}
\item[+]
a supervisor external interrupt (code~9) with the priority number
indicated by \z{vstopei};
\end{tightList}

\item
if \z{hstatus}.VGEIN is the valid number of a guest interrupt file,
bit~9 is one in both \z{vsip} and \z{vsie}, and \z{vstopei} is zero:
\begin{tightList}
\item[+]
a supervisor external interrupt (code~9) with priority number 256;
\end{tightList}

\item
if \z{hstatus}.VGEIN =~0, and \z{hvictl} fields IID =~9 and
$\mbox{IPRIO} \neq \mbox{0}$:
\begin{tightList}
\item[+]
a supervisor external interrupt (code~9) with priority number
\z{hvictl}.IPRIO;
\end{tightList}

\item
if \z{hvictl}.VTI =~0:
\begin{tightList}
\item[+]
all pending-and-enabled major interrupts indicated by
\z{vsip}/\z{vsiph} and \z{vsie}/\z{vsieh} other than a supervisor
external interrupt (code 9), with priority numbers assigned by
\z{hviprio1}/\z{hviprio1h} and \z{hviprio2}/\z{hviprio2h};
\end{tightList}

\item
if \z{hvictl} fields VTI =~1 and $\mbox{IID} \neq \mbox{9}$:
\begin{tightList}
\item[+]
the major interrupt specified by \z{hvictl} fields IID and IPRIO.
\end{tightList}

\end{itemize}

In the list above, all ``supervisor'' external interrupts are virtual,
directed to VS level, having major code~9 at VS level.

\begin{commentary}
The list of candidate interrupts can be reduced to two finalists
relatively easily by observing that the first three list items are
mutually exclusive of one another, and the remaining two items are also
mutually exclusive of one another.
\end{commentary}

\begin{commentary}
When both \z{hstatus}.VGEIN =~0 and \z{hvictl}.VTI =~1, the
absence of an interrupt for VS level can be indicated only by setting
\z{hvictl} fields IID =~9 and IPRIO~=~0.
Software might want to use the pair IID =~9, IPRIO~=~0 generally to
represent\/ \emph{no interrupt} in \z{hvictl}.
\end{commentary}

When no interrupt candidates satisfy the conditions of the list above,
\z{vstopi} is zero.
Else, \z{vstopi} fields IID and IPRIO are determined by the
highest-priority interrupt from among the candidates.
The usual priority order for supervisor level applies, as specified by
Table~\ref{tab:intrPrios-S} on page~\pageref{tab:intrPrios-S}, except
that priority numbers are taken from the candidate list above, not from
the supervisor-level \z{iprio} array.
As always, ties in nominal priority are broken by the default priority
order from Section~\ref{sec:majorIntrs}.
If bit IPRIOM (IPRIO Mode) of \z{hvictl} is zero,
IPRIO in \z{vstopi} is~1;
else, if the priority number for the highest-priority candidate is within the
range 1 to 255, IPRIO is that value;
else, IPRIO is set to either 0 or 255 in the manner documented
for \z{stopi} in Section~\ref{sec:stopi}.

%- - - - - - - - - - - - - - - - - - - - - - - - - - - - - - - - - - - -
\subsection{Interrupt traps to \mbox{VS-mode}}

Whenever the value of CSR \z{vstopi} is not zero, if either the current
privilege mode is \mbox{VS-mode} and the SIE bit in CSR \z{vsstatus}
is set, or the current privilege mode is \mbox{VU-mode}, a trap is
taken to \mbox{VS-mode} for the interrupt indicated by field IID of
\z{vstopi}.

The Exception Code field of \z{vscause} must implement at least as
many bits as needed to represent the largest value that field IID of
\z{vstopi} can have for the given hart.

